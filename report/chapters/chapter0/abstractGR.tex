\begin{center}
  \centering

  \vspace{0.5cm}
  \centering
  \textbf{\Large{Περίληψη}}
  \phantomsection
  \addcontentsline{toc}{section}{Περίληψη}

  \vspace{1cm}

\end{center}

Τα τελευταία χρόνια έχει παρατηρηθεί ραγδαία εξέλιξη του κλάδου της ρομποτικής και ιδιαίτερα αυτού των Συστήμάτων μη Επανδρωμένων Αεροσκαφών (ΣμηΕΑ). Η χρήση τους, ενώ αρχικά ήταν κυρίως στρατιωτική, αυξάνεται συνεχώς σε πληθώρα εφαρμογών, όπως η εξερεύνηση δυσπρόσιτων περιοχών, η κινηματογράφηση εκδηλώσεων, η αυτοματοποιημένη απογραφή προϊόντων κ.α. 

Η χρήση των μη επανδρωμένων αεροσκαφών σε κλειστούς χώρους απαιτεί πολύ καλή αντίληψη του περιβάλλοντος, άμεση απόκριση σε μεταβολές αυτού και συνεπώς αξιόπιστη εκτίμηση της θέσης τους μέσα στον χώρο. Για την επίτευξη αυτών των στόχων απαιτείται η χρήση όλων των αισθητήρων που διαθέτει το ρομπότ. Η πλοήγηση του drone με βέλτιστο τρόπο μέσα στο χώρο απαιτεί την ύπαρξη ενός εκ των προτέρων γνωστού πλάνου πτήσης το οποίο θα οδηγήσει στην επίτευξη του στόχου.

Η παρούσα διπλωματική εργασία εστιάζει στην επίλυση του προβλήματος της αυτόνομης και συνεχούς απογραφής προϊόντων σε οποιονδήποτε γνωστό χώρο. Με την χρήση των drones η διαδικασία αυτή απλουστεύεται με επιθυμητό αποτέλεσμα τον προσδιορισμό της θέσης των προϊόντων με ακρίβεια μερικών εκατοστών. Το πρόβλημα αυτό αποτελείται από δύο υπο-προβλήματα: α) αυτό του εντοπισμού θέσης στον κλειστό χώρο, και β) αυτό της πλήρους κάλυψης του χώρου αυτόνομα.     

Για την αντιμετώπιση των παραπάνω προβλημάτων χρησιμοποιείται γνωστός τρισδιάστατος χάρτης μορφής OctoMap. Κατά τη διάρκεια της έρευνας, υλοποιήθηκε ένας αλγόριθμος που βασίζεται σε \emph{φίλτρο σωματιδίων} (Particle Filter) και αξιοποιεί ένα σύνολο αποστάσεων περιμετρικά του drone για τον υπολογισμό της θέσης. Η πλοήγηση στον χώρο πραγματοποιείται με την χρήση ενός PID ελεγκτή θέσης και εξασφαλίζει την κίνηση με αποφυγή των γνωστών στατικών εμποδίων. Για την πλήρη κάλυψη του χώρου αρχικά πραγματοποιείται μια επιλογή των σημείων που πρέπει να διασχίσει το ρομπότ και στην συνέχεια με την ένωση αυτών δημιουργείται το τελικό μονοπάτι.

Τέλος, πραγματοποιήθηκε μια σειρά πειραμάτων τα οποία αρχικά εξετάζουν την αξιοπιστία του συστήματος εντοπισμού θέσης σε τρία είδη κινήσεων, καθώς και με διαφορετικές ταχύτητες σε κάθε μία από αυτές τις περιπτώσεις. Ταυτόχρονα, εξετάστηκαν διάφοροι τρόποι κίνησης στο χώρο, πραγματοποιώντας κάλυψη του χώρου με χρήση διαφορετικών χαρακτηριστικών του αισθητήρα ανά περίπτωση. Τα πειράματα αυτά πραγματοποιήθηκαν εξ' ολοκλήρου σε περιβάλλον προσομοίωσης.

\begin{flushright}
  \vspace{2cm}
  Κοσμάς Τσιάκας
  \\
  ktsiakas@ece.auth.gr
  \\
  Τμήμα Ηλεκτρολόγων Μηχανικών \& Μηχανικών Υπολογιστών
  \\
  Αριστοτέλειο Πανεπιστήμιο Θεσσαλονίκης
  \\
  Ιούνιος 2019
\end{flushright}