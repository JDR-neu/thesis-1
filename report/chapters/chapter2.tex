\chapter{Επισκόπηση της Ερευνητικής Περιοχής}
\label{chapter:sota}

Τα προβλήματα του εντοπισμού θέσης, της αυτόνομης πλοήγησης και της πλήρης κάλυψης χώρου από μη επανδρωμένα αεροσκάφη έχουν απασχολήσει έντονα την επιστημονική κοινότητα τα τελευταία χρόνια. Πληθώρα δημοσιεύσεων αναφέρουν διαφορετικές μεθόδους που προσπαθούν να επιλύσουν τα προαναφερθέντα προβλήματα. Οι λύσεις που έχουν παρουσιαστεί διαφέρουν κατά κύριο λόγο στον τύπο των αισθητήρων που χρησιμοποιούνται από τα drones, συνεπώς και την πληροφορία που είναι διαθέσιμη. Αυτό που εισάγει πολλούς περιορισμούς και διαφέρει από αντίστοιχες λύσεις για επίγεια ρομπότ είναι το βάρος το οποίο μπορούν να μεταφέρουν και η κατανάλωση ενέργειας που πρέπει να παραμένει χαμηλή, ώστε να μην μειώνεται σημαντικά η διάρκεια της πτήσης και το διαφορετικό κινηματικό μοντέλο.\\ 

Στη συνέχεια, θα εξετάσουμε ξεχωριστά για κάθε πρόβλημα τις λύσεις που έχουν προταθεί από άλλους ερευνητές.

\section{Πειράματα εντοπισμού θέσης}
\label{section:localization_tests}

Σκοπός των πειραμάτων αυτών είναι ο υπολογισμός της απόκλισης της εκτίμησης θέσης από την πραγματική. Για το λόγο αυτό, χρειαζόμαστε την πραγματική θέση (ground truth) του ρομπότ. Αυτό παρέχεται με την χρήση του plugin GazeboRosP3D\footnote{\href{http://docs.ros.org/electric/api/gazebo\_plugins/html/group\_\_GazeboRosP3D.html}{http://docs.ros.org/electric/api/gazebo\_plugins/html/group\_\_GazeboRosP3D.html}} στον προσομοιωτή Gazebo, το οποίο εξάγει την πληροφορία αυτή στο ROS topic $/ground\_truth/state$. Η εκτίμηση της θέσης από το σύστημα που δημιουργήθηκε αποστέλλεται στο ROS topic $/amcl\_pose$. \\
    
Ο αλγόριθμος εξετάστηκε σε τρία είδη κινήσεων:
\begin{itemize}
    \item {Ευθεία κίνηση}
    \item {Κίνηση σε σπιράλ}
    \item {Κίνηση σε μαίανδρο}
\end{itemize}

Για κάθε μία από αυτές τις περιπτώσεις εκτελέστηκαν πέντε διαφορετικές δοκιμές, με τρεις διαφορετικές ταχύτητες κίνησης του drone σε καθεμία. Σε όλα τα πειράματα, χρησιμοποιήθηκε το 100\% του αριθμού των σωματιδίων για την εξαγωγή της τελικής θέσης, ενώ οι παράμετροι του αλγορίθμου παρέμειναν σταθεροί και οι τιμές τους παρουσιάζονται στον \autoref{tab:mcl_params}. Δεν υπάρχει κάποιος συγκεκριμένος τρόπος για τον υπολογισμό των τιμών αυτών και για το λόγο αυτό επιλέχθηκαν μετά από αρκετές δοκιμές. Καθώς όμως γνωρίζουμε κάποια βασικά χαρακτηριστικά των περιβαλλόντων που μπορεί να χρησιμοποιηθεί το σύστημα εντοπισμού θέσης, δόθηκε μεγαλύτερη έμφαση σε μετρήσεις που βρίσκονται εντός του εύρους του αισθητήρα απόστασης σε γνωστά εμπόδια, και λιγότερη σε πολύ κοντινές μετρήσεις και απροσδόκητα εμπόδια.

\begin{table}[H]
    \begin{center}
        \caption{Τιμές παραμέτρων αλγορίθμου εντοπισμού θέσης}
        \label{tab:mcl_params}
        \begin{tabular}{| c | c | c | c | c | c | c |}
        \hline
        \rowcolor{Gray}
        Αριθμός σωματιδίων & $z_{hit}$ & $z_{short}$ & $z_{rand}$ & $z_{max}$ & laser $\sigma_{hit}$ & laser $\lambda_{short}$ \\
        300 & 0.6 & 0.1 & 0.1 & 0.2 & 0.02 & 0.1 \\
        \hline
        \end{tabular}
    \end{center}
\end{table}


Τα δεδομένα που χρησιμοποιούνται είναι της μορφής:
\begin{table}[H]
    \begin{center}
        \caption{Μορφή δεδομένων για εξαγωγή αποτελεσμάτων του localization}
        \label{tab:localization_data}
        \begin{tabular}{| c | c | c | c | c |}
        \hline
        \rowcolor{Gray}
        Timestamp & Error in x & Error in y & Error in z & Error in yaw \\
        ... & ... & ... & ... & ... \\
        \hline
        \end{tabular}
    \end{center}
\end{table}

Από τα δεδομένα αυτά, υπολογίστηκε το \emph{Absolute Position Error} και με την χρήση αυτού υπολογίστηκαν οι παρακάτω μετρικές για το  Σφάλμα θέσης (σε μέτρα) και προσανατολισμού (σε rad).

\begin{center}
    $Absolute\: Position\: Error = \sqrt{{error\_x}^2 + {error\_y}^2 + {error\_z}^2}$ 
\end{center}

\begin{itemize}
    \item{$Mean = {\frac {1}{n}}\left(\sum _{i=1}^{n}{error_{i}}\right)={\frac {error_{1}+error_{2}+\cdots +error_{n}}{n}}$ }
    \item{$Median = {\frac {error_{\lfloor (\#n+1)\div 2\rfloor }+error_{\lceil (\#n+1)\div 2\rceil }}{2}}$}
    \item{$Min\: Error$}
    \item{$Max\: Error$}
    \item{$Root\: Mean\: Square\: Error = {\sqrt{\frac {\sum _{i=1}^{N}({\hat {error}}_{i})^{2}}{N}}}$}
    \item{$Sum\: of\: Squared\: Error = \sum _{i=1}^{n} error_{i}^{2}$}
    \item{$Standard\: Deviation = {\sqrt {{\frac {1}{N-1}}\sum _{i=1}^{N}(error_{i}-{\bar {error}})^{2}}}$}
\end{itemize}

\subsection{Κίνηση σε ευθεία γραμμή}
\label{subsection:localization_tests_line}
Η κίνηση σε ευθεία γραμμή εξετάστηκε σε δύο διαφορετικούς χώρους, έναν διάδρομο και μία αποθήκη. Το drone έπρεπε να φτάσει σε ένα συγκεκριμένο ύψος και στη συνέχεια να διασχίσει τρία σημεία που δημιουργούσαν μία ευθεία στο χώρο. Η κίνηση αυτή στον διάδρομο, φαίνεται στο \autoref{fig:corridor_line}, ενώ στην αποθήκη στο \autoref{fig:warehouse_line}.

\begin{figure}[!ht]
        \begin{subfigure}{0.5\textwidth}
            \includegraphics[width=0.9\textwidth]{./images/chapter5/corridor_line.png}
                \caption{Διάδρομος}
             \label{fig:corridor_line}
        \end{subfigure}
        \begin{subfigure}{0.5\textwidth}
            \includegraphics[width=0.9\textwidth]{./images/chapter5/warehouse_line.png}
            \caption{Αποθήκη}
            \label{fig:warehouse_line}
        \end{subfigure}
        \caption{Κίνηση σε ευθεία γραμμή}
\end{figure} 

Το περιβάλλον του διαδρόμου περιέχει αρκετά συμμετρικά χαρακτηριστικά, κάτι που πιθανόν να οδηγήσει σε μεγαλύτερο σφάλμα. Αντίθετα, η αποθήκη παρουσιάζει μεγαλύτερο ενδιαφέρον, καθώς πρόκειται για ένα πιο ρεαλιστικό περιβάλλον. Όπως προαναφέρθηκε, εξετάστηκαν τρεις διαφορετικές ταχύτητες κίνησης του ρομπότ σε κάθε χώρο. Τα αποτελέσματα των πειραμάτων αυτών, φαίνονται αναλυτικά στους παρακάτω πίνακες.


\iffalse % TABLE TEMPLATE STRUCTURE%
\begin{table}[H]
    \begin{center}
        \centering
        \caption{..............}
        \label{tab:...}
        \begin{tabular}{| c | c | c | c | c | c | c | c | }
        \hline
        \rowcolor{Gray}
        \# & Mean & Median & Min & Max & SSE & STD & RMSE \\
        \hline
        1 & ... & ... & ... & ...& ... & ... & ... \\
        2 & ... & ... & ... & ...& ... & ... & ... \\
        3 & ... & ... & ... & ...& ... & ... & ... \\
        4 & ... & ... & ... & ...& ... & ... & ... \\
        5 & ... & ... & ... & ...& ... & ... & ... \\
        \hline
        \textbf{Avg.} & ... & ... & ... & ... & ... & ... & ... \\
        \hline
        \end{tabular}
    \end{center}
\end{table}
\fi

\begin{table}[H]
    \centering
    \caption{Σφάλμα θέσης (σε μέτρα) για κίνηση σε ευθεία γραμμή με χαμηλή ταχύτητα σε διάδρομο}
    \label{tab:position_error_line_slow_corridor}
    \begin{tabular}{| c | c | c | c | c | c | c | c | }
        \hline
        \rowcolor{Gray}
        \# & Mean & Median & Min & Max & SSE & STD & RMSE \\
        \hline
        1 & 0.13894 & 0.14003 & 0.01624 & 0.19890 & 19.18185 & 0.02638 & 0.14142 \\
        2 & 0.15154 & 0.13033 & 0.01023 & 0.30975 & 27.85205 & 0.06365 & 0.16436 \\
        3 & 0.14150 & 0.13463 & 0.01266 & 0.34980 & 23.42868 & 0.06351 & 0.15509 \\
        4 & 0.08683 & 0.07937 & 0.00766 & 0.19531 & 8.85282 & 0.04161 & 0.09628 \\ 
        5 & 0.11425 & 0.11648 & 0.00958 & 0.24263 & 14.23755 & 0.03994 & 0.12102 \\
        \hline
        \textbf{Avg.} & 0.12661 & 0.12017 & 0.01127 & 0.25928 & 18.71059 & 0.04702 & 0.13563 \\
        \hline
    \end{tabular}
\end{table}

\begin{table}[H]
    \begin{center}
        \centering
        \caption{Σφάλμα θέσης (σε μέτρα) για κίνηση σε ευθεία γραμμή με κανονική ταχύτητα σε διάδρομο}
        \label{tab:position_error_line_normal_corridor}
        \begin{tabular}{| c | c | c | c | c | c | c | c | }
        \hline
        \rowcolor{Gray}
        \# & Mean & Median & Min & Max & SSE & STD & RMSE \\
        1 & 0.16242 & 0.17036 & 0.00997 & 0.32717 & 15.20109 & 0.07639 & 0.17945 \\
        2 & 0.22075 & 0.26521 & 0.00718 & 0.38309 & 27.18928 & 0.09102 & 0.23874 \\
        3 & 0.08396 & 0.07386 & 0.00522 & 0.21196 & 4.34510 & 0.04520 & 0.09534 \\
        4 & 0.11701 & 0.12045 & 0.01269 & 0.33264 & 10.98280 & 0.08984 & 0.14747 \\
        5 & 0.15474 & 0.15491 & 0.01836 & 0.35479 & 13.49755 & 0.06558 & 0.16804 \\
        \hline
        \textbf{Avg.} & 0.14778 & 0.15696 & 0.01068 & 0.32193 & 14.24316 & 0.07361 & 0.16581 \\
        \hline
        \end{tabular}
    \end{center}
\end{table}

\begin{table}[H]
    \begin{center}
        \centering
        \caption{Σφάλμα θέσης (σε μέτρα) για κίνηση σε ευθεία γραμμή με υψηλή ταχύτητα σε διάδρομο}
        \label{tab:position_error_line_fast_corridor}
        \begin{tabular}{| c | c | c | c | c | c | c | c | }
        \hline
        \rowcolor{Gray}
        \# & Mean & Median & Min & Max & SSE & STD & RMSE \\
        \hline
        1 & 0.24806 & 0.21986 & 0.01206 & 0.57617 & 32.85699 & 0.14016 & 0.28483 \\
        2 & 0.17828 & 0.11405 & 0.00874 & 0.47371 & 17.97368 & 0.12724 & 0.21893 \\ 
        3 & 0.19176 & 0.13981 & 0.01011 & 0.58964 & 21.81091 & 0.12578 & 0.22925 \\
        4 & 0.33894 & 0.28538 & 0.00933 & 0.73504 & 72.62943 & 0.23017 & 0.40956 \\
        5 & 0.18411 & 0.12833 & 0.01144 & 0.56394 & 20.48029 & 0.13074 & 0.22571 \\
        \hline
        \textbf{Avg.} & 0.22823 & 0.17749 & 0.01033 & 0.58770 & 33.15026 & 0.15082 & 0.27366 \\
        \hline
        \end{tabular}
    \end{center}
\end{table}

\begin{table}[H]
    \begin{center}
        \centering
        \caption{Σφάλμα θέσης (σε μέτρα) για κίνηση σε ευθεία γραμμή με χαμηλή ταχύτητα σε αποθήκη}
        \label{tab:position_error_line_slow_warehouse}
        \begin{tabular}{| c | c | c | c | c | c | c | c | }
        \hline
        \rowcolor{Gray}
        \# & Mean & Median & Min & Max & SSE & STD & RMSE \\
        \hline
        1 & 0.114325 & 0.111956 & 0.007847 & 0.330900 & 16.779918 & 0.054259 & 0.126536 \\ 
        2 & 0.187610 & 0.215719 & 0.015321 & 0.395761 & 43.740688 & 0.100231 & 0.212681 \\
        3 & 0.126385 & 0.100733 & 0.015605 & 0.269795 & 19.686598 & 0.065937 & 0.142536 \\ 
        4 & 0.082295 & 0.086740 & 0.009050 & 0.156036 & 7.810667 & 0.034388 & 0.089184 \\ 
        5 & 0.116021 & 0.111720 & 0.011625 & 0.246022 & 14.842384 & 0.040495 & 0.122878 \\
        \hline
        \textbf{Avg.} & 0.125327 & 0.125374 & 0.011890 & 0.279703 & 20.572051 & 0.059062 & 0.138763 \\
        \hline
        \end{tabular}
    \end{center}
\end{table}

\begin{table}[H]
    \begin{center}
        \centering
        \caption{Σφάλμα θέσης (σε μέτρα) για κίνηση σε ευθεία γραμμή με κανονική ταχύτητα σε αποθήκη}
        \label{tab:position_error_line_normal_warehouse}
        \begin{tabular}{| c | c | c | c | c | c | c | c | }
        \hline
        \rowcolor{Gray}
        \# & Mean & Median & Min & Max & SSE & STD & RMSE \\
        \hline
        1 & 0.208280 & 0.199181 & 0.017934 & 0.544024 & 28.269758 & 0.125675 & 0.243191 \\ 
        2 & 0.115554 & 0.120715 & 0.004416 & 0.354717 & 11.331192 & 0.075929 & 0.138233 \\
        3 & 0.092544 & 0.077934 & 0.002976 & 0.295298 & 5.014900 & 0.047663 & 0.104074 \\
        4 & 0.168933 & 0.153410 & 0.016503 & 0.348709 & 18.885880 & 0.115212 & 0.204409 \\
        5 & 0.088185 & 0.091700 & 0.009636 & 0.213946 & 4.842245 & 0.049877 & 0.101287 \\
        \hline
        \textbf{Avg.} & 0.134699 & 0.128588 & 0.010293 & 0.351339 & 13.668795 & 0.082871 & 0.158238 \\
        \hline
        \end{tabular}
    \end{center}
\end{table}

\begin{table}[H]
    \begin{center}
        \centering
        \caption{Σφάλμα θέσης (σε μέτρα) για κίνηση σε ευθεία γραμμή με υψηλή ταχύτητα σε αποθήκη}
            \label{tab:position_error_line_fast_warehouse}
        \begin{tabular}{| c | c | c | c | c | c | c | c | }
        \hline
        \rowcolor{Gray}
        \# & Mean & Median & Min & Max & SSE & STD & RMSE \\
        \hline
        1 & 0.136676 & 0.074564 & 0.006855 & 0.395738 & 11.975489 & 0.110164 & 0.175457 \\ 
        2 & 0.154984 & 0.092472 & 0.010285 & 0.450830 & 15.579335 & 0.121538 & 0.196862 \\
        3 & 0.223545 & 0.203554 & 0.016012 & 0.461632 & 23.246522 & 0.090357 & 0.241073 \\
        4 & 0.286846 & 0.314360 & 0.016945 & 0.590272 & 46.801999 & 0.095643 & 0.302341 \\
        5 & 0.182972 & 0.104353 & 0.005001 & 0.627469 & 24.213108 & 0.169823 & 0.249488 \\
        \hline
        \textbf{Avg.} & 0.197004 & 0.157861 & 0.011020 & 0.505188 & 24.363290 & 0.117505 & 0.233044 \\
        \hline
        \end{tabular}
    \end{center}
\end{table}

\begin{figure}[!ht]
    \centering
    \includegraphics[width=1.00\textwidth]{./images/chapter5/corridor_slow_line_4.png}
    \caption{Οπτικοποίηση καλύτερης πορείας για κίνηση σε ευθεία γραμμή σε διάδρομο}
    \label{fig:path_line_slow_corridor}
\end{figure}

\begin{figure}[!ht]
    \centering
    \includegraphics[width=1.00\textwidth]{./images/chapter5/warehouse_slow_line_4.png}
    \caption{Οπτικοποίηση καλύτερης πορείας για κίνηση σε ευθεία γραμμή σε αποθήκη}
    \label{fig:path_line_slow_warehouse}
\end{figure}

Για όλες τις παραπάνω περιπτώσεις έχει υπολογιστεί και το σφάλμα προσανατολισμού του drone. Δεν αναφέρεται όμως αναλυτικά, καθώς σε όλες τις δοκιμές που πραγματοποιήθηκαν, ανεξαρτήτως του περιβάλλοντος και της ταχύτητας, το μέσο σφάλμα είναι συνεχώς μικρότερο του $0.1$ rad. Αυτό προφανώς οφείλεται στο γεγονός το ρομπότ δεν πραγματοποιεί καμία περιστροφή κατά την κίνηση του και η κατεύθυνσή του λαμβάνεται απευθείας από τον αισθητήρα IMU που διαθέτει. Στο \autoref{fig:path_line_slow_corridor} και στο \autoref{fig:path_line_slow_warehouse}, όπου παρέχεται η οπτικοποίηση της πραγματικής και της εκτιμώμενης πορείας στον χώρο, το μεγαλύτερο σφάλμα εμφανίζεται ως προς τον $y$ άξονα, με αποτέλεσμα να δημιουργείται μια αστάθεια στην ευθεία κίνηση του ρομπότ.

Όπως παρατηρούμε, το μικρότερο σφάλμα παρουσιάζεται κατά την αργή κίνηση του ρομπότ και στις δύο περιπτώσεις. Εκτός της μέσης τιμής, χαμηλή παραμένει και η μέγιστη τιμή που μπορεί να αποκτήσει το σφάλμα, σε αντίθεση με την περίπτωση της υψηλής ταχύτητας. Εκεί όπου το σφάλμα παρουσιάζει και μεγαλύτερη διασπορά, το μέγιστο σφάλμα είναι αρκετά υψηλό για να εγγυηθεί ασφαλή πλοήγηση του drone στο χώρο.

\clearpage

\subsection{Κίνηση σε σπιράλ}
\label{subsection:localization_tests_spiral}

Στη συνέχεια, εξετάστηκε η κίνηση σε πορεία μορφής σπιράλ. Κατά την κίνηση αυτή, το drone καλείται να κινηθεί κυκλικά και προς τα πάνω, καθώς κάθε σημείο βρίσκεται ψηλότερα από το προηγούμενό του. Κατά τη διάρκεια της κίνησης, ο προσανατολισμός του drone παραμένει σταθερός και ίδιος και με τον αρχικό. Επομένως, τα επόμενα πειράματα μελετούν τη συμπεριφορά του συστήματος, καθώς το drone μεταβάλλει τις συντεταγμένες $x$, $y$ και $z$ ταυτόχρονα. Οι χώροι που χρησιμοποιήθηκαν ήταν ένας κενός χώρος, χωρίς καθόλου εμπόδια παρά μόνο τοίχους και η αποθήκη που αναφέρθηκε και προηγουμένως, όπως φαίνεται στο \autoref{fig:spiral}.

\begin{figure}[!ht]
        \begin{subfigure}{0.5\textwidth}
            \includegraphics[width=0.9\textwidth]{./images/chapter5/box_spiral.png}
                \caption{Κενός χώρος}
             \label{fig:box_spiral}
        \end{subfigure}
        \begin{subfigure}{0.5\textwidth}
            \includegraphics[width=0.9\textwidth]{./images/chapter5/warehouse_spiral.png}
            \caption{Αποθήκη}
            \label{fig:warehouse_spiral}
        \end{subfigure}
        \caption{Κίνηση σε σπιράλ}
        \label{fig:spiral}
\end{figure} 

Παρακάτω παρουσιάζεται το σφάλμα που υπήρξε σε κάθε διαφορετική περίπτωση. Το σφάλμα του προσανατολισμού δεν παρουσιάζεται για το περιβάλλον του κενού χώρου, καθώς παραμένει μικρότερο του $0.1$ rad για κάθε δοκιμή. Αντίθετα, κατά την κίνηση στον χώρο της αποθήκης, το σφάλμα μεταβάλλεται και παρουσιάζεται παρακάτω.

\begin{table}[H]
    \centering
    \caption{Σφάλμα θέσης (σε μέτρα) για κίνηση σε σπιράλ με χαμηλή ταχύτητα σε κενό χώρο}
    \label{tab:position_error_spiral_slow_box}
    \begin{tabular}{| c | c | c | c | c | c | c | c | }
        \hline
        \rowcolor{Gray}
        \# & Mean & Median & Min & Max & SSE & STD & RMSE \\
        \hline
        1 & 0.26975 & 0.31742 & 0.01130 & 0.54886 & 112.97555 & 0.15513 & 0.31114 \\
        2 & 0.11821 & 0.12610 & 0.00211 & 0.35319 & 23.34022 & 0.06880 & 0.13676 \\
        3 & 0.16422 & 0.17058 & 0.00408 & 0.49477 & 45.33603 & 0.09678 & 0.19060 \\
        4 & 0.09561 & 0.08395 & 0.01105 & 0.31162 & 16.14039 & 0.06017 & 0.11296 \\
        5 & 0.08542 & 0.09533 & 0.00544 & 0.26510 & 12.17983 & 0.04609 & 0.09706 \\
        \hline
        \textbf{Avg.} & 0.14664 & 0.15868 & 0.00680 & 0.39471 & 41.99440 & 0.08539 & 0.16970 \\
        \hline
    \end{tabular}
\end{table}

\begin{table}[H]
    \begin{center}
        \centering
        \caption{Σφάλμα θέσης (σε μέτρα) για κίνηση σε σπιράλ με κανονική ταχύτητα σε κενό χώρο}
        \label{tab:position_error_spiral_normal_box}
        \begin{tabular}{| c | c | c | c | c | c | c | c | }
        \hline
        \rowcolor{Gray}
        \# & Mean & Median & Min & Max & SSE & STD & RMSE \\
        1 & 0.10616 & 0.07093 & 0.01113 & 0.41376 & 14.79245 & 0.10061 & 0.14621 \\
        2 & 0.15549 & 0.13576 & 0.00313 & 0.52998 & 35.06631 & 0.14126 & 0.21002 \\
        3 & 0.10490 & 0.08614 & 0.00346 & 0.42966 & 14.19397 & 0.08920 & 0.13766 \\
        4 & 0.12569 & 0.08250 & 0.00794 & 0.39558 & 17.18100 & 0.08748 & 0.15310 \\
        5 & 0.10362 & 0.08400 & 0.00438 & 0.36648 & 12.58463 & 0.08410 & 0.13342 \\
        \hline
        \textbf{Avg.} & 0.11917 & 0.09186 & 0.00601 & 0.42709 & 18.76367 & 0.10053 & 0.15608 \\
        \hline
        \end{tabular}
    \end{center}
\end{table}

\begin{table}[H]
    \begin{center}
        \centering
        \caption{Σφάλμα θέσης (σε μέτρα) για κίνηση σε σπιράλ με υψηλή ταχύτητα σε κενό χώρο}
        \label{tab:position_error_spiral_fast_box}
        \begin{tabular}{| c | c | c | c | c | c | c | c | }
        \hline
        \rowcolor{Gray}
        \# & Mean & Median & Min & Max & SSE & STD & RMSE \\
        \hline
        1 & 0.13799 & 0.14837 & 0.00192 & 0.55425 & 22.96128 & 0.12062 & 0.18322 \\
        2 & 0.14768 & 0.10509 & 0.01162 & 0.59520 & 27.27895 & 0.13518 & 0.20014 \\
        3 & 0.21043 & 0.14093 & 0.00524 & 0.60197 & 52.90884 & 0.17706 & 0.27493 \\
        4 & 0.11989 & 0.10549 & 0.00683 & 0.39358 & 16.66327 & 0.09719 & 0.15429 \\
        5 & 0.10156 & 0.06495 & 0.01010 & 0.49884 & 13.70723 & 0.09663 & 0.14014 \\
        \hline
        \textbf{Avg.} & 0.14351 & 0.11297 & 0.00714 & 0.52877 & 26.70392 & 0.12533 & 0.19054 \\
        \hline
        \end{tabular}
    \end{center}
\end{table}

\begin{table}[H]
    \begin{center}
        \centering
        \caption{Σφάλμα θέσης (σε μέτρα) για κίνηση σε σπιράλ με χαμηλή ταχύτητα σε αποθήκη}
        \label{tab:position_error_spiral_slow_warehouse}
        \begin{tabular}{| c | c | c | c | c | c | c | c | }
        \hline
        \rowcolor{Gray}
        \# & Mean & Median & Min & Max & SSE & STD & RMSE \\
        \hline
        1 & 0.25256 & 0.14320 & 0.00475 & 1.28265 & 273.42973 & 0.30430 & 0.39539 \\
        2 & 0.18359 & 0.19685 & 0.01235 & 0.72798 & 61.80427 & 0.11271 & 0.21540 \\
        3 & 0.14626 & 0.11277 & 0.00233 & 1.29595 & 84.01740 & 0.17188 & 0.22565 \\
        4 & 0.16322 & 0.14987 & 0.00663 & 1.23738 & 73.79179 & 0.16350 & 0.23099 \\
        5 & 0.22972 & 0.19084 & 0.01103 & 0.63412 & 113.64658 & 0.17994 & 0.29176 \\
        \hline
        \textbf{Avg.} & 0.19507 & 0.15871 & 0.00742 & 1.03562 & 121.33796 & 0.18646 & 0.27184 \\
        \hline
        \end{tabular}
    \end{center}
\end{table}

\begin{table}[H]
    \begin{center}
        \centering
        \caption{Σφάλμα προσανατολισμού (σε rad) για κίνηση σε σπιράλ με χαμηλή ταχύτητα σε αποθήκη}
        \label{tab:orientation_error_spiral_slow_warehouse}
        \begin{tabular}{| c | c | c | c | c | c | c | c | }
        \hline
        \rowcolor{Gray}
        \# & Mean & Median & Min & Max & SSE & STD & RMSE \\
        \hline
        1 & 0.60339 & 0.04200 & 0.00003 & 6.21851 & 5990.57539 & 1.75008 & 1.85071 \\
        2 & 0.06974 & 0.04814 & 0.00001 & 6.20961 & 196.46472 & 0.37780 & 0.38405 \\
        3 & 1.08731 & 0.03621 & 0.00009 & 6.25999 & 9651.88863 & 2.16106 & 2.41860 \\
        4 & 0.05675 & 0.05289 & 0.00005 & 3.96727 & 29.33249 & 0.13417 & 0.14563 \\
        5 & 0.15260 & 0.10726 & 0.00024 & 6.11451 & 470.37756 & 0.57384 & 0.59358 \\
        \hline
        \textbf{Avg.} & 0.39396 & 0.05730 & 0.00008 & 5.75398 & 3267.72776 & 0.99939 & 1.07851 \\
        \hline
        \end{tabular}
    \end{center}
\end{table}

\begin{table}[H]
    \begin{center}
        \centering
        \caption{Σφάλμα θέσης (σε μέτρα) για κίνηση σε σπιράλ με κανονική ταχύτητα σε αποθήκη}
        \label{tab:position_error_spiral_normal_warehouse}
        \begin{tabular}{| c | c | c | c | c | c | c | c | }
        \hline
        \rowcolor{Gray}
        \# & Mean & Median & Min & Max & SSE & STD & RMSE \\
        \hline
        1 & 0.15150 & 0.13640 & 0.00369 & 0.34122 & 31.77614 & 0.10388 & 0.18366 \\
        2 & 0.16504 & 0.17102 & 0.00390 & 0.44813 & 34.05847 & 0.10344 & 0.19474 \\
        3 & 0.18469 & 0.17301 & 0.00221 & 0.48730 & 43.33518 & 0.10551 & 0.21268 \\
        4 & 0.19868 & 0.15884 & 0.01203 & 0.60410 & 48.55944 & 0.14970 & 0.24871 \\
        5 & 0.10943 & 0.08234 & 0.00127 & 0.47569 & 19.55174 & 0.10302 & 0.15025 \\
        \hline
        \textbf{Avg.} & 0.16187 & 0.14432 & 0.00462 & 0.47129 & 35.45619 & 0.11311 & 0.19801 \\
        \hline
        \end{tabular}
    \end{center}
\end{table}

\begin{table}[H]
    \begin{center}
        \centering
        \caption{Σφάλμα προσανατολισμού (σε rad) για κίνηση σε σπιράλ με κανονική ταχύτητα σε αποθήκη}
        \label{tab:orientation_error_spiral_normal_warehouse}
        \begin{tabular}{| c | c | c | c | c | c | c | c | }
        \hline
        \rowcolor{Gray}
        \# & Mean & Median & Min & Max & SSE & STD & RMSE \\
        \hline
        1 & 0.04635 & 0.01585 & 0.00010 & 5.43492 & 85.80986 & 0.29839 & 0.30181 \\
        2 & 0.03806 & 0.03612 & 0.00000 & 0.56901 & 2.00954 & 0.02810 & 0.04730 \\
        3 & 0.13571 & 0.02801 & 0.00018 & 6.22394 & 578.76217 & 0.76572 & 0.77726 \\
        4 & 3.20807 & 6.17517 & 0.00252 & 6.23041 & 15512.83529 & 3.07926 & 4.44539 \\
        5 & 0.02982 & 0.01975 & 0.00072 & 2.42426 & 6.87179 & 0.08398 & 0.08907 \\
        \hline
        \textbf{Avg.} & 0.69160 & 1.25498 & 0.00070 & 4.17651 & 3237.25773 & 0.85109 & 1.13217 \\
        \hline
        \end{tabular}
    \end{center}
\end{table}

\begin{table}[H]
    \begin{center}
        \centering
        \caption{Σφάλμα θέσης (σε μέτρα) για κίνηση σε σπιράλ με υψηλή ταχύτητα σε αποθήκη}
                \label{tab:position_error_spiral_fast_warehouse}
        \begin{tabular}{| c | c | c | c | c | c | c | c | }
        \hline
        \rowcolor{Gray}
        \# & Mean & Median & Min & Max & SSE & STD & RMSE \\
        \hline
        1 & 0.14558 & 0.14035 & 0.00234 & 0.37128 & 23.15082 & 0.10030 & 0.17675 \\
        2 & 0.16669 & 0.16334 & 0.00234 & 0.53938 & 44.57385 & 0.13661 & 0.21547 \\
        3 & 0.31894 & 0.30145 & 0.00691 & 0.76302 & 99.71369 & 0.19338 & 0.37292 \\
        4 & 0.17109 & 0.16619 & 0.01282 & 0.39827 & 37.36532 & 0.08499 & 0.19102 \\
        5 & 0.22178 & 0.20365 & 0.01477 & 0.57958 & 64.42152 & 0.12949 & 0.25678 \\
        \hline
        \textbf{Avg.} & 0.20482 & 0.19500 & 0.00784 & 0.53031 & 53.84504 & 0.12895 & 0.24259 \\
        \hline
        \end{tabular}
    \end{center}
\end{table}

\begin{table}[H]
    \begin{center}
        \centering
        \caption{Σφάλμα προσανατολισμού (σε rad) για κίνηση σε σπιράλ με υψηλή ταχύτητα σε αποθήκη}
        \label{tab:orientation_error_spiral_fast_warehouse}
        \begin{tabular}{| c | c | c | c | c | c | c | c | }
        \hline
        \rowcolor{Gray}
        \# & Mean & Median & Min & Max & SSE & STD & RMSE \\
        \hline
        1 & 0.01975 & 0.01055 & 0.00012 & 0.12255 & 0.63240 & 0.02153 & 0.02921 \\
        2 & 1.79168 & 0.10750 & 0.00105 & 6.23073 & 10366.41819 & 2.75611 & 3.28608 \\
        3 & 0.02055 & 0.01369 & 0.00013 & 0.13945 & 0.49767 & 0.01649 & 0.02634 \\
        4 & 0.13325 & 0.04807 & 0.00000 & 6.20004 & 437.62583 & 0.64032 & 0.65373 \\
        5 & 0.05985 & 0.04065 & 0.00027 & 6.20379 & 106.40434 & 0.32470 & 0.33001 \\
        \hline
        \textbf{Avg.} & 0.40501 & 0.04409 & 0.00031 & 3.77931 & 2182.31569 & 0.75183 & 0.86507 \\
        \hline
        \end{tabular}
    \end{center}
\end{table}

\begin{figure}[!ht]
    \centering
    \includegraphics[width=1.00\textwidth]{./images/chapter5/box_slow_spiral_5.png}
    \caption{Οπτικοποίηση καλύτερης πορείας για κίνηση σε σπιράλ πορεία σε κενό χώρο}
    \label{fig:path_spiral_slow_corridor}
\end{figure}

\begin{figure}[!ht]
    \centering
    \includegraphics[width=1.00\textwidth]{./images/chapter5/warehouse_normal_spiral_5.png}
    \caption{Οπτικοποίηση καλύτερης πορείας για κίνηση σε σπιράλ πορεία σε αποθήκη}
    \label{fig:path_spiral_normal_warehouse}
\end{figure}
Όπως φαίνεται στην αναπαράσταση της πορείας που ακολούθησε το ρομπότ, πρόκειται για μία αρκετά σταθερή κίνηση παρόλο που υπάρχει συνεχή μεταβολή της κίνησης σε κάθε άξονα. Πιο συγκεκριμένα, η κίνηση με κανονική ταχύτητα επιφέρει το χαμηλότερο σφάλμα από όλες τις περιπτώσεις. Η κίνηση με αργή ταχύτητα επίσης οδηγεί σε χαμηλό σφάλμα, εκτός από μία μόνο περίπτωση. Επίσης, παρατηρείται ότι το σφάλμα προσανατολισμού λαμβάνει υψηλή μέγιστη τιμή παρά τη σχετικά μικρή μέση τιμή του. Αυτό σημαίνει ότι στιγμιαία μπορεί η εκτίμηση να είναι αρκετά λανθασμένη, όμως το συνολικό σύστημα εντοπισμού θέσης δεν επηρεάζεται.

\clearpage

\subsection{Κίνηση σε μαίανδρο}
\label{subsection:localization_tests_meander}

Η τελευταία περίπτωση που εξετάστηκε είναι η κίνηση σε πορεία σχήματος μαιάνδρου. Το drone καλείται να ακολουθήσει ένα σύνολο σημείων το οποίο φαίνεται στο \autoref{fig:meander} για κάθε περιβάλλον. Η σειρά διάσχισης είναι από τα χαμηλά και εξωτερικά σημεία προς αυτά που βρίσκονται σε μεγαλύτερο ύψος και στο εσωτερικού του μονοπατιού. Επίσης, στην κίνηση αυτή ο προσανατολισμούς του ρομπότ αλλάζει ανάλογα με την περίπτωση, στοχεύοντας κάθε φορά σε επόμενο σημείο.

Στη συνέχεια, παρουσιάζονται τα αποτελέσματα της σύγκρισης ανάμεσα στην εκτιμώμενη και στην πραγματική θέση, καθώς και στο σφάλμα προσανατολισμού για κάθε μία ξεχωριστή περίπτωση. Η διαφορά με τις προηγούμενες κινήσεις είναι ότι η κατεύθυνση του drone μεταβάλλεται συχνά.

\begin{figure}[!ht]
        \begin{subfigure}{0.5\textwidth}
            \includegraphics[width=0.9\textwidth]{./images/chapter5/box_meander.png}
                \caption{Κενός χώρος}
             \label{fig:box_meander}
        \end{subfigure}
        \begin{subfigure}{0.5\textwidth}
            \includegraphics[width=0.9\textwidth]{./images/chapter5/warehouse_meander.png}
            \caption{Αποθήκη}
            \label{fig:warehouse_meander}
        \end{subfigure}
        \caption{Κίνηση σε μαίανδρο}
        \label{fig:meander}
\end{figure}


\begin{table}[H]
    \centering
    \caption{Σφάλμα θέσης (σε μέτρα) για κίνηση σε μαίανδρο με χαμηλή ταχύτητα σε κενό χώρο}
    \label{tab:position_error_meander_slow_box}
    \begin{tabular}{| c | c | c | c | c | c | c | c | }
        \hline
        \rowcolor{Gray}
        \# & Mean & Median & Min & Max & SSE & STD & RMSE \\
        \hline
        1 & 0.13941 & 0.13917 & 0.01053 & 0.40395 & 88.71098 & 0.07046 & 0.15620 \\
        2 & 0.17959 & 0.18069 & 0.00112 & 0.31979 & 120.20220 & 0.06296 & 0.19031 \\
        3 & 0.25041 & 0.20634 & 0.00875 & 0.51618 & 319.34917 & 0.16220 & 0.29834 \\
        4 & 0.12573 & 0.09644 & 0.01047 & 0.36000 & 78.24017 & 0.08151 & 0.14984 \\
        5 & 0.15761 & 0.14391 & 0.00233 & 0.33064 & 115.69076 & 0.07992 & 0.17671 \\
        \hline
        \textbf{Avg.} & 0.17055 & 0.15331 & 0.00664 & 0.38611 & 144.43866 & 0.09141 & 0.19428 \\
        \hline
    \end{tabular}
\end{table}

\begin{table}[H]
    \centering
    \caption{Σφάλμα προσανατολισμού (σε rad) για κίνηση σε μαίανδρο με χαμηλή ταχύτητα σε κενό χώρο}
    \label{tab:orientation_error_meander_slow_box}
    \begin{tabular}{| c | c | c | c | c | c | c | c | }
        \hline
        \rowcolor{Gray}
        \# & Mean & Median & Min & Max & SSE & STD & RMSE \\
        \hline
        1 & 0.04816 & 0.02652 & 0.00007 & 6.23700 & 312.04984 & 0.28901 & 0.29295 \\
        2 & 0.03423 & 0.02254 & 0.00051 & 6.20982 & 123.86660 & 0.19016 & 0.19319 \\
        3 & 0.59045 & 0.02264 & 0.00002 & 6.24419 & 12486.34445 & 1.76982 & 1.86548 \\
        4 & 0.03180 & 0.02515 & 0.00025 & 6.25226 & 120.58698 & 0.18330 & 0.18602 \\
        5 & 0.03221 & 0.01813 & 0.00002 & 6.24227 & 159.83427 & 0.20522 & 0.20770 \\
        \hline
        \textbf{Avg.} & 0.14737 & 0.02300 & 0.00017 & 6.23711 & 2640.53643 & 0.52750 & 0.54907 \\
        \hline
    \end{tabular}
\end{table}

\begin{table}[H]
    \begin{center}
        \centering
        \caption{Σφάλμα θέσης (σε μέτρα) για κίνηση σε μαίανδρο με κανονική ταχύτητα σε κενό χώρο}
        \label{tab:position_error_meander_normal_box}
        \begin{tabular}{| c | c | c | c | c | c | c | c | }
        \hline
        \rowcolor{Gray}
        \# & Mean & Median & Min & Max & SSE & STD & RMSE \\
        1 & 0.13383 & 0.13209 & 0.01411 & 0.37084 & 49.13156 & 0.06705 & 0.14968 \\
        2 & 0.16226 & 0.16357 & 0.00549 & 0.42715 & 70.75790 & 0.09190 & 0.18647 \\
        3 & 0.17202 & 0.17970 & 0.00107 & 0.40352 & 76.20777 & 0.09507 & 0.19653 \\
        4 & 0.24672 & 0.22460 & 0.00168 & 0.62291 & 156.15264 & 0.15479 & 0.29124 \\
        5 & 0.20597 & 0.19350 & 0.00964 & 0.61865 & 112.45581 & 0.11872 & 0.23772 \\
        \hline
        \textbf{Avg.} & 0.18416 & 0.17869 & 0.00640 & 0.48862 & 92.94114 & 0.10551 & 0.21233 \\
        \hline
        \end{tabular}
    \end{center}
\end{table}

\begin{table}[H]
    \centering
    \caption{Σφάλμα προσανατολισμού (σε rad) για κίνηση σε μαίανδρο με κανονική ταχύτητα σε κενό χώρο}
    \label{tab:orientation_error_meander_normal_box}
    \begin{tabular}{| c | c | c | c | c | c | c | c | }
        \hline
        \rowcolor{Gray}
        \# & Mean & Median & Min & Max & SSE & STD & RMSE \\
        \hline
        1 & 0.04689 & 0.02533 & 0.00000 & 6.21171 & 190.95459 & 0.29140 & 0.29508 \\
        2 & 0.03398 & 0.01843 & 0.00002 & 6.22261 & 115.54487 & 0.23591 & 0.23828 \\
        3 & 0.04552 & 0.02767 & 0.00002 & 6.19883 & 154.07015 & 0.27578 & 0.27944 \\
        4 & 0.03950 & 0.02718 & 0.00000 & 6.23749 & 156.12471 & 0.28860 & 0.29121 \\
        5 & 0.03897 & 0.02222 & 0.00007 & 6.23997 & 101.45287 & 0.22246 & 0.22579 \\
        \hline
        \textbf{Avg.} & 0.04097 & 0.02417 & 0.00002 & 6.22212 & 143.62944 & 0.26283 & 0.26596 \\
        \hline
    \end{tabular}
\end{table}

\begin{table}[H]
    \begin{center}
        \centering
        \caption{Σφάλμα θέσης (σε μέτρα) για κίνηση σε μαίανδρο με υψηλή ταχύτητα σε κενό χώρο}
        \label{tab:position_error_meander_fast_box}
        \begin{tabular}{| c | c | c | c | c | c | c | c | }
        \hline
        \rowcolor{Gray}
        \# & Mean & Median & Min & Max & SSE & STD & RMSE \\
        \hline
        1 & 0.41997 & 0.47342 & 0.01095 & 0.97306 & 378.97266 & 0.21126 & 0.47008 \\
        2 & 0.24618 & 0.26350 & 0.00640 & 0.64563 & 132.31387 & 0.15444 & 0.29058 \\
        3 & 0.51469 & 0.59397 & 0.00974 & 1.07424 & 575.06789 & 0.25805 & 0.57572 \\
        4 & 0.21217 & 0.21110 & 0.00669 & 0.58169 & 100.03961 & 0.14052 & 0.25446 \\
        5 & 0.24979 & 0.22645 & 0.01368 & 0.72454 & 153.89782 & 0.16209 & 0.29774 \\
        \hline
        \textbf{Avg.} & 0.32856 & 0.35369 & 0.00949 & 0.79983 & 268.05837 & 0.18527 & 0.37772 \\
        \hline
        \end{tabular}
    \end{center}
\end{table}

\begin{table}[H]
    \centering
    \caption{Σφάλμα προσανατολισμού (σε rad) για κίνηση σε μαίανδρο με υψηλή ταχύτητα σε κενό χώρο}
    \label{tab:orientation_error_meander_fast_box}
    \begin{tabular}{| c | c | c | c | c | c | c | c | }
        \hline
        \rowcolor{Gray}
        \# & Mean & Median & Min & Max & SSE & STD & RMSE \\
        \hline
        1 & 0.13294 & 0.02812 & 0.00016 & 6.21965 & 739.34580 & 0.64317 & 0.65659 \\
        2 & 0.52434 & 0.05197 & 0.00013 & 6.26766 & 4538.56307 & 1.61959 & 1.70186 \\
        3 & 0.03920 & 0.02077 & 0.00007 & 6.23783 & 86.16475 & 0.21944 & 0.22285 \\
        4 & 0.37290 & 0.03904 & 0.00093 & 6.25117 & 3154.05689 & 1.37972 & 1.42880 \\
        5 & 0.07434 & 0.03889 & 0.00009 & 6.23467 & 302.51878 & 0.41089 & 0.41745 \\
        \hline
        \textbf{Avg.} & 0.22874 & 0.03576 & 0.00027 & 6.24219 & 1764.12986 & 0.85456 & 0.88551 \\
        \hline
    \end{tabular}
\end{table}

\begin{table}[H]
    \begin{center}
        \centering
        \caption{Σφάλμα θέσης (σε μέτρα) για κίνηση σε μαίανδρο με χαμηλή ταχύτητα σε αποθήκη}
        \label{tab:position_error_meander_slow_warehouse}
        \begin{tabular}{| c | c | c | c | c | c | c | c | }
        \hline
        \rowcolor{Gray}
        \# & Mean & Median & Min & Max & SSE & STD & RMSE \\
        \hline
        1 & 0.16461 & 0.15817 & 0.00502 & 0.30393 & 98.66991 & 0.05284 & 0.17289 \\
        2 & 0.12103 & 0.12106 & 0.00141 & 0.25082 & 55.44583 & 0.04189 & 0.12807 \\
        3 & 0.07707 & 0.07404 & 0.00402 & 0.19366 & 25.75932 & 0.04161 & 0.08758 \\
        4 & 0.20393 & 0.17800 & 0.00716 & 0.46433 & 162.21094 & 0.08732 & 0.22184 \\
        5 & 0.24893 & 0.25953 & 0.00073 & 0.55382 & 277.37072 & 0.14133 & 0.28625 \\
        \hline
        \textbf{Avg.} & 0.16312 & 0.15816 & 0.00367 & 0.35331 & 123.89134 & 0.07300 & 0.17933 \\
        \hline
        \end{tabular}
    \end{center}
\end{table}

\begin{table}[H]
    \centering
    \caption{Σφάλμα προσανατολισμού (σε rad) για κίνηση σε μαίανδρο με χαμηλή ταχύτητα σε αποθήκη}
    \label{tab:orientation_error_meander_slow_warehouse}
    \begin{tabular}{| c | c | c | c | c | c | c | c | }
        \hline
        \rowcolor{Gray}
        \# & Mean & Median & Min & Max & SSE & STD & RMSE \\
        \hline
        1 & 0.03500 & 0.02523 & 0.00001 & 4.66208 & 62.37786 & 0.13295 & 0.13746 \\
        2 & 0.04165 & 0.02371 & 0.00002 & 4.72697 & 108.26666 & 0.17408 & 0.17897 \\
        3 & 0.02163 & 0.01688 & 0.00004 & 4.28138 & 23.19017 & 0.08024 & 0.08310 \\
        4 & 0.03641 & 0.02727 & 0.00003 & 3.62527 & 19.71341 & 0.06823 & 0.07733 \\
        5 & 0.04105 & 0.04252 & 0.00001 & 3.66151 & 20.75406 & 0.06668 & 0.07830 \\
        \hline
        \textbf{Avg.} & 0.03515 & 0.02712 & 0.00002 & 4.19144 & 46.86043 & 0.10444 & 0.11103 \\
        \hline
    \end{tabular}
\end{table}

\begin{table}[H]
    \begin{center}
        \centering
        \caption{Σφάλμα θέσης (σε μέτρα) για κίνηση σε μαίανδρο με κανονική ταχύτητα σε αποθήκη}
        \label{tab:position_error_meander_normal_warehouse}
        \begin{tabular}{| c | c | c | c | c | c | c | c | }
        \hline
        \rowcolor{Gray}
        \# & Mean & Median & Min & Max & SSE & STD & RMSE \\
        \hline
        1 & 1.07495 & 0.82528 & 0.00200 & 2.27038 & 4097.95301 & 0.67114 & 1.26719 \\
        2 & 0.50567 & 0.46140 & 0.01199 & 1.67076 & 742.84486 & 0.24718 & 0.56283 \\
        3 & 0.29637 & 0.25446 & 0.00681 & 1.53723 & 255.81494 & 0.20987 & 0.36313 \\
        4 & 0.09909 & 0.09806 & 0.00625 & 0.20110 & 17.51989 & 0.04845 & 0.11030 \\
        5 & 0.32056 & 0.29371 & 0.00209 & 1.35330 & 281.32884 & 0.19527 & 0.37533 \\
        \hline
        \textbf{Avg.} & 0.27797 & 0.25026 & 0.00762 & 1.05732 & 270.44227 & 0.16022 & 0.32147\\
        \hline
        \end{tabular}
    \end{center}
\end{table}

\begin{table}[H]
    \centering
    \caption{Σφάλμα προσανατολισμού (σε rad) για κίνηση σε μαίανδρο με κανονική ταχύτητα σε αποθήκη}
    \label{tab:orientation_error_meander_normal_warehouse}
    \begin{tabular}{| c | c | c | c | c | c | c | c | }
        \hline
        \rowcolor{Gray}
        \# & Mean & Median & Min & Max & SSE & STD & RMSE \\
        \hline
        1 & 0.17421 & 0.04715 & 0.00027 & 6.21773 & 814.83564 & 0.73556 & 0.75565 \\
        2 & 0.05018 & 0.03114 & 0.00007 & 6.22709 & 154.26887 & 0.25158 & 0.25649 \\
        3 & 0.04360 & 0.02087 & 0.00001 & 6.22664 & 141.97265 & 0.26705 & 0.27052 \\
        4 & 0.31922 & 0.01610 & 0.00002 & 6.23409 & 2608.81174 & 1.30804 & 1.34598 \\
        5 & 0.46441 & 0.08796 & 0.00041 & 6.20350 & 4753.61433 & 1.47166 & 1.54285 \\
        \hline
        \textbf{Avg.} & 0.21032 & 0.04064 & 0.00016 & 6.22181 & 1694.70065 & 0.80678 & 0.83430 \\
        \hline
    \end{tabular}
\end{table}

\begin{table}[H]
    \begin{center}
        \centering
        \caption{Σφάλμα θέσης (σε μέτρα) για κίνηση σε μαίανδρο με υψηλή ταχύτητα σε αποθήκη}
            \label{tab:position_error_meander_fast_warehouse}
        \begin{tabular}{| c | c | c | c | c | c | c | c | }
        \hline
        \rowcolor{Gray}
        \# & Mean & Median & Min & Max & SSE & STD & RMSE \\
        \hline
        1 & 0.31894 & 0.28612 & 0.01336 & 1.40791 & 205.45110 & 0.15796 & 0.35590 \\
        2 & 0.38288 & 0.28412 & 0.00728 & 1.47429 & 383.08797 & 0.28306 & 0.47610 \\
        3 & 0.25022 & 0.20078 & 0.01576 & 1.45133 & 166.62539 & 0.19783 & 0.31894 \\
        4 & 0.42412 & 0.34258 & 0.00868 & 1.40271 & 428.05702 & 0.26686 & 0.50105 \\
        5 & 0.36361 & 0.29345 & 0.00946 & 1.76602 & 336.18949 & 0.27287 & 0.45456 \\
        \hline
        \textbf{Avg.} & 0.34795 & 0.28141 & 0.01091 & 1.50045 & 303.88219 & 0.23572 & 0.42131 \\
        \hline
        \end{tabular}
    \end{center}
\end{table}

\begin{table}[H]
    \centering
    \caption{Σφάλμα προσανατολισμού (σε rad) για κίνηση σε μαίανδρο με υψηλή ταχύτητα σε αποθήκη}
    \label{tab:orientation_error_meander_fast_warehouse}
    \begin{tabular}{| c | c | c | c | c | c | c | c | }
        \hline
        \rowcolor{Gray}
        \# & Mean & Median & Min & Max & SSE & STD & RMSE \\
        \hline
        1 & 0.58158 & 0.07151 & 0.00175 & 6.23028 & 5223.65858 & 1.69825 & 1.79458 \\
        2 & 0.03930 & 0.02466 & 0.00002 & 6.23386 & 76.62195 & 0.20933 & 0.21293 \\
        3 & 0.03897 & 0.02281 & 0.00007 & 6.23772 & 78.94221 & 0.21611 & 0.21953 \\
        4 & 0.10915 & 0.09875 & 0.00012 & 6.23123 & 137.97221 & 0.26277 & 0.28447 \\
        5 & 0.03319 & 0.02453 & 0.00002 & 5.71171 & 44.32420 & 0.16173 & 0.16505 \\
        \hline
        \textbf{Avg.} & 0.16044 & 0.04845 & 0.00040 & 6.12896 & 1112.30383 & 0.50964 & 0.53531 \\
        \hline
    \end{tabular}
\end{table}

\begin{figure}[!ht]
    \centering
    \includegraphics[width=1.00\textwidth]{./images/chapter5/box_slow_meander_4.png}
    \caption{Οπτικοποίηση καλύτερης πορείας για κίνηση σε πορεία μαιάνδρου σε κενό χώρο}
    \label{fig:path_meander_slow_corridor}
\end{figure}

\begin{figure}[!ht]
    \centering
    \includegraphics[width=1.00\textwidth]{./images/chapter5/warehouse_slow_meander_1.png}
    \caption{Οπτικοποίηση καλύτερης πορείας για κίνηση σε πορεία μαιάνδρου σε αποθήκη}
    \label{fig:path_meander_slow_warehouse}
\end{figure}

Η περίπτωση αυτή παρουσιάζει το μεγαλύτερο σφάλμα από όλες τις προηγούμενες κινήσεις που εξετάστηκαν. Και πάλι, όμως, η χαμηλή ταχύτητα οδηγεί στο χαμηλότερο σφάλμα θέσης και προσανατολισμού. Αξίζει να σημειωθεί η μεγάλη διασπορά του σφάλματος κατά την κίνηση σε υψηλή ταχύτητα, το οποίο σημαίνει ότι πρόκειται για μια ασταθή κίνηση. 
\section{Πλήρης Κάλυψη Χώρου}
\label{section:coverage}

Η πλήρης κάλυψη ενός χώρου πρόκειται για το πρόβλημα της δημιουργίας ενός μονοπατιού το οποίο διέρχεται από όλα τα σημεία ενδιαφέροντος ενός περιβάλλοντος, καθώς παράλληλα γίνεται αποφυγή εμποδίων. Σύμφωνα με το \cite{galceran2013}, οι έξι απαιτήσεις της διαδικασίας πλήρους κάλυψης χώρου είναι οι εξής:
\begin{itemize}

    \item{Το ρομπότ πρέπει να διασχίσει όλα τα σημεία στην περιοχή ενδιαφέροντος, καλύπτοντας την πλήρως}
    \item{Το ρομπότ πρέπει να καλύπτει την περιοχή, χωρίς την ύπαρξη αλληλοεπικαλυπτόμενων διαδρομών}
    \item{Απαιτούνται συνεχείς και διαδοχικές διεργασίες, χωρίς την επανάληψη καμίας τροχιάς}
    \item{Το ρομπότ πρέπει να αποφεύγει κάθε είδους εμπόδιο}
    \item{Απλοϊκές τροχιές (π.χ. ευθείες ή κυκλικές κινήσεις) θα έπρεπε να χρησιμοποιούνται, καθώς προσφέρουν απλότητα στην κίνηση}
    \item{Ένα βέλτιστο μονοπάτι προτιμάται, εφόσον αυτό είναι εφικτό}

\end{itemize}

Υπάρχουν δύο κατηγορίες προσεγγίσεων για το θέμα αυτό, η απόλυτη και η ευριστική. Κατά την πρώτη, ο χώρος διαχωρίζεται σε τμήματα και μπορεί να εγγυηθεί την πλήρη κάλυψη του χώρου, ενώ κατά την ευριστική μέθοδο το ρομπότ ακολουθεί ένα σύνολο απλών κανόνων που επηρεάζουν την κίνηση του και πιθανόν να μην οδηγήσουν με επιτυχία στην πλήρη κάλυψη.

Στο \cite{7496385} παρουσιάζεται μία απόλυτη προσέγγιση στο πρόβλημα της πλήρους κάλυψης χώρου για UAVs. Ο χώρος χωρίζεται σε πολυγωνικά τμήματα με μέγεθος που επηρεάζεται από το πεδίο όρασης (Field of View) της κάμερας. Το κέντρο κάθε πολυγώνου θεωρείται ως το σημείο που πρέπει να βρεθεί το drone, ώστε να καλυφθεί η περιοχή γύρω του. Στη συνέχεια, χρησιμοποιείται ένας wavefront propagation αλγόριθμος για να βρει όλα τα δυνατά μονοπάτια που μπορούν να ενώσουν τα σημεία αυτά. Για κάθε ένα μονοπάτι υπολογίζεται το συνολικό του κόστος και επιλέγεται αυτό με το μικρότερο δυνατό. Στο τέλος, η πορεία ομαλοποιείται χρησιμοποιώντας πολυωνυμικές συναρτήσεις. Ένα σημαντικό χαρακτηριστικό της προσέγγισης αυτής είναι ότι κατά την δημιουργία των μονοπατιών, λαμβάνεται υπόψη η περιστροφική κίνηση που απαιτείται από το drone για να μεταβεί στο επόμενο σημείο, καθώς αυτή αυξάνει το χρόνο εκτέλεσης και είναι επιθυμητή η ελαχιστοποίηση των στροφών κατά την κίνηση στο χώρο.

Οι Bircher κ.α. \cite{bircher2015} παρουσιάζουν έναν αλγόριθμο για την τρισδιάστατη κάλυψη χώρου που στοχεύει στην αυτόνομη διαδικασία της επιθεώρησης χρησιμοποιώντας μη επανδρωμένα αεροσκάφη. Στην προσέγγιση αυτή, για κάθε σημείο του χάρτη υπολογίζουμε τη θέση η οποία επιφέρει την καλύτερη δυνατή θέαση του σημείου αυτού, με βάση τα χαρακτηριστικά των αισθητήρων που διαθέτει το ρομπότ. Στην συνέχεια, εφαρμόζεται μια επαναληπτική διαδικασία πεπερασμένων βημάτων, μέσω της οποίας συνδέονται τα σημεία που ελαχιστοποιούν το κόστος, είτε αυτό είναι η απόσταση είτε ο χρόνος εκτέλεσης. Η καλύτερη δυνατή σύνδεση των σημείων επιτυγχάνεται μέσω της επίλυσης του προβλήματος του περιοδεύοντος εμπόρου (Travelling Salesman Problem).

Μια αρκετά διαφορετική προσέγγιση παρουσιάζεται στο \cite{1262545}. Πιο συγκεκριμένα, χρησιμοποιούνται νευρωνικά δίκτυα για την πλήρη κάλυψη στατικού και μεταβαλλόμενου χώρου σε πραγματικό χρόνο. Η υλοποίηση αυτή εφαρμόζεται για χάρτη δύο διαστάσεων, μπορεί να καλύψει την ύπαρξη ενός ή περισσοτέρων ρομποτικών πρακτόρων και παρουσιάζει αξιόπιστα αποτελέσματα. Ο χώρος διακριτοποιείται με βάση το μέγεθος του ρομπότ και τα σημεία που προκύπτουν ενώνονται μέσω των νευρώνων του δικτύου. Η δυναμική μορφή του δικτύου μπορεί να μεταβάλλει στιγμιαία τα μονοπάτια που δημιουργούνται με βάση την προσθήκη ή την αφαίρεση εμποδίων.

Οι ευριστικές προσεγγίσεις βασίζονται στην κάλυψη του χώρου με διάφορα μοτίβα, όπως για παράδειγμα υλοποιείται στο \cite{DiFranco2016}. Στην περίπτωση αυτή, παρουσιάζεται ένας αλγόριθμος ο οποίος δημιουργεί ένα μονοπάτι που ικανοποιεί, εκτός των άλλων, και τους ενεργειακούς περιορισμούς του ρομπότ. Μετά την δημιουργία μιας πορείας που αποτελείται κυρίως από κινήσεις προς τα μπροστά και πίσω, εξετάζεται η ταχύτητα με την οποία πρέπει να κινείται το drone, ώστε να ελαχιστοποιείται η κατανάλωση ενέργειας.


