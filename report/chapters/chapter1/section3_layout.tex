\section{Διάρθρωση της Αναφοράς}
\label{section:layout}

Η διάρθρωση της παρούσας διπλωματικής εργασίας είναι η εξής:

\begin{itemize}
  \item{\textbf{Κεφάλαιο \ref{chapter:sota}:}
      Γίνεται ανασκόπησή της ερευνητικής περιοχής που αφορά την εύρεση θέσης μη επανδρωμένων αεροσκαφών σε γνωστό χώρο και την πλήρη κάλυψη του χώρου από αυτό.
    }
  \item{\textbf{Κεφάλαιο \ref{chapter:theory}:} Περιγράφονται τα βασικά θεωρητικά στοιχεία
  στα οποία βασίστηκαν οι υλοποιήσεις. Πιο συγκεκριμένα, περιγράφεται η αρχή λειτουργίας των μη επανδρωμένων αεροσκαφών και ο τρόπος λειτουργίας ενός PID ελεγκτή.
    }
  \item{\textbf{Κεφάλαιο \ref{chapter:tools}:} Περιγράφονται τα εργαλεία που χρησιμοποιήθηκαν για την υλοποίηση του συστήματος εντοπισμού θέσης και πλήρης κάλυψης χώρου. Πιο συγκεκριμένα, περιγράφεται το μεσολειτουργικό σύστημα ROS, πάνω στο οποίο βασίστηκε όλη η υλοποίηση, το σύνολο ROS πακέτων Hector Quadrotor καθώς και οι βιβλιοθήκες OctoMap, Particle Filter και Open Motion Planning Library.
    }
  \item{\textbf{Κεφάλαιο \ref{chapter:implementations}:} Πλήρης περιγραφή των υλοποιήσεων των αλγορίθμων εύρεσης θέσης, πλοήγησης και πλήρης κάλυψης χώρου. 
    }
  \item{\textbf{Κεφάλαιο \ref{chapter:experiments}:} Παρουσιάζεται αναλυτικά η μεθοδολογία των
      πειραμάτων και τα αποτελέσματα.
    }
  \item{\textbf{Κεφάλαιο \ref{chapter:conclusions}:} Παρουσιάζονται τα τελικά συμπεράσματα.
    }
  \item{\textbf{Κεφάλαιο \ref{chapter:future_work}:} Αναφέρονται τα
      προβλήματα που προέκυψαν και προτείνονται θέματα για μελλοντική
      μελέτη, αλλαγές και επεκτάσεις.
    }
\end{itemize}