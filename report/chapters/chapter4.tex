\chapter{Υλοποιήσεις}
\label{chapter:implementations}

Ο στόχος της παρούσας διπλωματικής εργασίας είναι η δημιουργία ενός συστήματος που θα είναι ικανό να εντοπίζει την θέση ενός drone σε γνωστό τρισδιάστατο χάρτη και να δημιουργεί κατάλληλα και ασφαλή μονοπάτια ώστε να καλυφθεί κάθε σημείο αυτού. Επομένως, είναι απαραίτητο να έχουμε εκ των προτέρων διαθέσιμο τον εκάστοτε τρισδιάστατο χάρτη του περιβάλλοντος. Προτιμήθηκε η μορφή OctoMap, καθώς περιέχει σημαντικές δυνατότητες που διευκολύνουν την διαδικασία του localization.

Σε όλες τις υλοποιήσεις χρησιμοποιήθηκε το drone που φαίνεται στο \autoref{fig:drone}.

\begin{figure}[!ht]
    \centering
    \includegraphics[width=0.9\textwidth]{./images/chapter4/drone.png}
    \caption{Το μοντέλο του drone που χρησιμοποιήθηκε} 
    \label{fig:drone}
\end{figure}

Το Quadcopter αυτό αποτελείται από τα παρακάτω τμήματα:
\begin{itemize}
        \item {Έναν αισθητήρα TeraRanger Tower\footnote{\href{https://www.terabee.com/shop/lidar-tof-multi-directional-arrays/teraranger-tower/}{https://www.terabee.com/shop/lidar-tof-multi-directional-arrays/teraranger-tower/}} που προσφέρει 8 ακτίνες laser περιμετρικά με εμβέλεια 14 μέτρων}
    \item{Έναν αισθητήρα laser για τον υπολογισμό του ύψους στο οποίο βρίσκεται το drone}
    \item{Έναν αισθητήρα IMU}
\end{itemize}

Στο πρώτο μέρος του συγκεκριμένου κεφαλαίου θα περιγραφεί ο αλγόριθμος Monte Carlo Localization, ο οποίος υλοποιήθηκε για τον εντοπισμό της θέσης του drone σε τρισδιάστατο περιβάλλον μορφής OctoMap. Στο δεύτερο μέρος θα αναλυθεί ο τρόπος υπολογισμού των σημείων που οδηγούν στην πλήρη κάλυψη του χώρου και η ένωση αυτών με ευριστικούς αλγορίθμους.

\section{Εντοπισμός θέσης}
\label{section:localization_impl}

Όπως αναφέρθηκε στο \autoref{section:libpf}, η κύρια μέθοδος που χρησιμοποιήθηκε για τον εντοπισμό της θέσης είναι ο αλγόριθμος \emph{Φίλτρου Σωματιδίων} (Particle Filter, PF). 

Η βασική ιδέα του φίλτρου είναι η αναπαράσταση της εκ των υστέρων πεποίθησης με ένα σύνολο τυχαίων δειγμάτων κατανομής που έχουν ληφθεί από την συγκεκριμένη εκ των υστέρων κατανομή. Κάθε σωματίδιο είναι ένα συνεπτυγμένο στιγμιότυπο της κατάστασης σε κάποια χρονική στιγμή. Ο αλγόριθμος αυτός απαιτεί ως είσοδο το προηγούμενο σύνολο σωματιδίων, ένα μοντέλο μέτρησης και ένα μοντέλο κίνησης. Η κύρια πηγή πληροφορίας για το πρώτο είναι οι αποστάσεις που προκύπτουν από τον αισθητήρα laser, ενώ για το δεύτερο είναι ο υπολογισμός της κίνησης του drone.

Το σύστημα δίνει τη δυνατότητα για την επίλυση του προβλήματος τοπικού και καθολικού εντοπισμού θέσης. Στην πρώτη περίπτωση, τα σωματίδια του φίλτρου αρχικοποιούνται γύρω από την αρχική θέση χρησιμοποιώντας μία κανονική κατανομή με δεδομένη τιμή της τυπικής απόκλισης, η οποία παρέχεται ως παράμετρος. Στην περίπτωση του Global Localization, τα σωματίδια αρχικοποιούνται με μία κανονική κατανομή σε όλο τον χάρτη του περιβάλλοντος. Η αρχική θέση παρέχεται επίσης ως παράμετρος του συστήματος. Η αρχικοποίηση των σωματιδίων φαίνεται στο \autoref{fig:initialize_particles}.

\begin{figure}[!ht]
    \centering
    \includegraphics[width=0.8\textwidth]{./images/chapter4/initialize_particles.png}
    \caption{Αρχικοποίηση σωματιδίων}
    \label{fig:initialize_particles}
\end{figure} 

Όσον αφορά το μοντέλο κίνησης του ρομπότ, σε αντίθεση με τα επίγεια οχήματα, τα οποία διαθέτουν αυτήν την πληροφορία μέσω αισθητήρων κίνησης (π.χ. κωδικοποιητές στους τροχούς), σε ένα drone αυτό δεν είναι εφικτό. Μία λύση θα ήταν η χρήση οπτικής οδομετρίας, μέσω της κάμερας που διαθέτει. Καθώς όμως η ανάπτυξη του συστήματος πραγματοποιείται σε περιβάλλον προσομοίωσης με ιδανικές επιφάνειες, δεν θα είχαμε τα επιθυμητά αποτελέσματα. Για το λόγο αυτό, συνδυάζουμε την πληροφορία που προκύπτει από το IMU, τον αισθητήρα ύψους και την ταχύτητα που του παρέχεται, ώστε να προκύψει μια εκτίμηση της κίνησης και της περιστροφής του drone. 

Για τον συγχρονισμό των μηνυμάτων χρησιμοποιείται το ROS πακέτο Message Filters\footnote{\href{http://wiki.ros.org/message\_filters}{http://wiki.ros.org/message\_filters}} και πιο συγκεκριμένα το ApproximateTime Policy. Η μέθοδος αυτή χρησιμοποιεί έναν αλγόριθμο που προσαρμόζεται ανάλογα με τις χρονικές σημάνσεις των μηνυμάτων που δέχεται. Για κάθε συνδυασμό μηνυμάτων που λαμβάνονται γίνονται οι υπολογισμοί που φαίνονται στον \autoref{alg:odometry_calculator}.

\begin{algorithm}[!ht]
 \caption{Αλγόριθμος υπολογισμού κίνησης και προσανατολισμού}
 \label{alg:odometry_calculator}
 \begin{algorithmic}[1]
    \Function{calculateOdometry}{height, imu, velocity}
        \State $dt \gets time.now - time.previous$
        \State $yaw \gets imu.yaw$
        \State $msg \gets NULL$ \Comment{Contains position and orientation that will be published}
        \State $msg.position.x = position.x + (cos(yaw) \cdot velocity.x - sin(yaw) \cdot velocity.y) * dt$
        \State $msg.position.y = position.y + (sin(yaw) \cdot velocity.x + cos(yaw) \cdot velocity.y) * dt$
        \State $msg.position.z = height$
        \State $msg.orientation = imu.orientation$ 
        \State $publish(msg)$
    \EndFunction 
    \end{algorithmic}
\end{algorithm}

Αρχικά, υπολογίζεται το χρονικό διάστημα $dt$ που μεσολάβησε ανάμεσα στις διαδοχικές κλήσεις της συνάρτησης. Από το μήνυμα του IMU υπολογίζεται η τιμή του $yaw$ στο drone. Δεδομένης αυτής της τιμής, περιστρέφουμε το διάνυσμα της ταχύτητας $velocity$, ώστε να μετασχηματιστεί από το πλαίσιο του ρομπότ στο πλαίσιο του κόσμου που βρίσκεται, όπως φαίνεται παρακάτω:
\begin{equation*}
     linear\_velocity_{transformed} = 
     \left[ {\begin{array}{ccc}
        cos\theta & sin\theta & 0 \\
        -sin\theta & cos\theta & 0 \\
        0 & 0 & 1
    \end{array} } \right]
    \left[ {\begin{array}{c}
        linear\_velocity_{x} \\
        linear\_velocity_{y} \\
        linear\_velocity_{z}
    \end{array} } \right]
\end{equation*}

Στη συνέχεια, η ποσότητα αυτή ολοκληρώνεται στο χρόνο που μεσολάβησε και προστίθεται στην προηγούμενη τιμή της θέσης. Το ύψος λαμβάνεται απευθείας από την τιμή $height$ του αισθητήρα ύψους. Τέλος, ο προσανατολισμός του drone δίνεται άμεσα από τον αισθητήρα IMU. Η σύνθεση αυτών αποστέλλεται στο κατάλληλο ROS topic, $/odom$, και η διαδικασία αυτή επαναλαμβάνεται συνεχώς.

Όσον αφορά το μοντέλο μέτρησης, αυτό χρησιμοποιεί τις μετρήσεις που προέρχονται από το laser, ώστε να εκτιμήσει το νέο βάρος κάθε σωματιδίου, ανάλογα με την θέση του. Αρχικά, οι μετρήσεις του αισθητήρα laser έρχονται σε 8 διαφορετικά ROS topic, τύπου sensor\_msgs/Range. Με την χρήση μιας τροποποιημένης έκδοσης του πακέτου teraranger\_array\_converter\footnote{\href{https://github.com/kosmastsk/teraranger\_array\_converter}{https://github.com/kosmastsk/teraranger\_array\_converter}} τα μηνύματα αυτά συγχωνεύονται σε ένα topic και στη συνέχεια μετατρέπονται σε μορφή sensor\_msgs/LaserScan. Η μορφή αυτή αποτελεί τον πιο συνηθισμένο τύπο για χειρισμό δεδομένων από αισθητήρα laser και περιέχει χρήσιμες πληροφορίες σχετικά με την ελάχιστη γωνία κάλυψης, τη γωνία μεταξύ διαδοχικών μετρήσεων και το εύρος του laser. Συνεπώς, ανακατασκευάζουμε τα σημεία στον τρισδιάστατο χώρο και μετατρέπουμε το μήνυμα σε μορφή PointCloud, της βιβλιοθήκης PCL\footnote{\href{http://pointclouds.org/}{http://pointclouds.org/}}. Το βήμα αυτό είναι χρήσιμο για την περίπτωση όπου το laser περιέχει μεγάλο αριθμό ακτινών και η επεξεργασία τους θα οδηγήσει σε αργοπορημένη απόκριση, δημιουργώντας την ανάγκη για υποδειγματοληψία. Στην συγκεκριμένη περίπτωση, δεν πραγματοποιείται καμία επεξεργασία, παρά μόνο παρέχεται μία λύση που καλύπτει κάθε αισθητήρα απόστασης, ανεξαρτήτως χαρακτηριστικών. Το τελευταίο βήμα της προ-επεξεργασίας είναι ο μετασχηματισμός των δεδομένων ως προς το κέντρο του κάθε σωματιδίου. Χρησιμοποιείται ο μετασχηματισμός μεταξύ του ρομπότ και του αισθητήρα, ο οποίος παρέχεται μέσω της κλάσης \emph{tf2\_ros::Buffer} της βιβλιοθήκης TF του ROS. Αυτός συνδυάζεται με την θέση του κάθε σωματιδίου και προκύπτει ο τελικός μετασχηματισμός, ώστε τα δεδομένα να βρίσκονται στην σωστή θέση.

\begin{figure}[!ht]
    \centering
    \includegraphics[width=0.8\textwidth]{./images/chapter4/pf.png}
        \caption{Διαδικασία ενός φίλτρου σωματιδίων} 
    \label{fig:pf}
\end{figure}

Εφόσον η προ-επεξεργασία έχει ολοκληρωθεί το φίλτρο ξεκινάει τη λειτουργία του, η οποία σχηματικά περιγράφεται στο \autoref{fig:pf}. Ο αριθμός των σωματιδίων ορίζεται ως παράμετρος στο σύστημα και είναι μια τιμή που επιτρέπει την ισορροπία ανάμεσα στην ακρίβεια των υπολογισμών και των υπολογιστικών πόρων που είναι διαθέσιμοι. Αρχικά, εφαρμόζεται δειγματοληψία των σωματιδίων ανάλογα με το βάρος τους. Στη συνέχεια, μετατοπίζουμε την θέση του κάθε σωματιδίου, σύμφωνα με το κινηματικό μοντέλο που έχει δημιουργηθεί. Μέσω της βιβλιοθήκης TF το σύστημα αναζητά την τρέχουσα θέση του ρομπότ στον κόσμο, την οποία μετασχηματίζει σε ένα διαφορετικό σύστημα συντεταγμένων, με χρήση της μεθόδου $transform(\cdot)$ της κλάσης \emph{tf2\_ros::Buffer}. Δίνοντας ως όρισμα το μοναδιαίο διάνυσμα του ρομπότ, επιστρέφει τη νέα θέση στο σύστημα συντεταγμένων που μεταβάλλεται με την μεταβολή της οδομετρίας. Έχοντας πλέον τις δύο αυτές θέσεις, μπορούμε να υπολογίσουμε την διαφορά τους και συνεπώς την κίνηση που έχει κάνει το ρομπότ. Ο μετασχηματισμός αυτός εφαρμόζεται σε όλα τα σωματίδια, ώστε να μετατοπιστούν στις νέες τους θέσεις.

Τελικό στάδιο του φίλτρου είναι η αξιολόγηση των μετρήσεων που έχουν ληφθεί από τον αισθητήρα απόστασης με βάση το μοντέλο μέτρησης. Το κάθε σωματίδιο αξιολογείται αναθέτοντας σε αυτό, ανάλογα με το μοντέλο που έχει καθοριστεί ένα βάρος. Το βάρος αυτό προσεγγίζει την πιθανότητα το συγκεκριμένο σωματίδιο να είναι η ορθή κατάσταση, δηλαδή η μέτρηση που έχει ληφθεί να αντιστοιχεί σε αυτήν που θα ανίχνευε αν όντως βρισκόταν σε αυτήν την κατάσταση. Εκτός από μία εκτίμηση θέσης από κάθε σωματίδιο, το σύστημα εξάγει και μια συνολική εκτίμηση για την θέση. Αυτή προκύπτει από ένα ποσοστό των καλύτερων σωματιδίων, το οποίο μπορεί να επιλεγεί από τον χρήστη. Δεν υπάρχει κάποια συγκεκριμένη τιμή του ποσοστού που να εγγυάται καλύτερα αποτελέσματα, καθώς αυτό μπορεί να διαφέρει ανάλογα με τις ιδιαιτερότητες κάθε περίπτωσης.

Ένα βασικό πρόβλημα των αισθητήρων είναι η ύπαρξη θορύβου, κάτι που συχνά δεν λαμβάνεται υπόψη στους υπολογισμούς. Στην συγκεκριμένη περίπτωση όμως, η έλλειψη ακρίβειας που παρουσιάζουν τα μοντέλα των αισθητήρων ενσωματώνεται στην μοντελοποίηση της διαδικασίας μέτρησης ως πυκνότητας μιας υπό συνθήκη πιθανότητας, αντί μιας αιτιοκρατικής συνάρτησης. Το μοντέλο μέτρησης που χρησιμοποιήθηκε περιγράφεται στο \cite{thrun2005} και περιλαμβάνει τέσσερις τύπους σφαλμάτων μέτρησης, που είναι όλοι βασικοί για την σωστή λειτουργία του μοντέλου: το λιγοστό θόρυβο μέτρησης, τα σφάλματα που οφείλονται σε μη αναμενόμενα αντικείμενα, τα σφάλματα που οφείλονται σε αποτυχίες εντοπισμού αντικειμένων και τον τυχαίο ανεξήγητο θόρυβο. Η πραγματική απόσταση $z_t^k$ που βρίσκεται η κατάσταση μπορεί να υπολογιστεί εύκολα, μέσω των συναρτήσεων της βιβλιοθήκης OctoMap που παρέχουν αυτήν την δυνατότητα του Raytracing. Σε κάθε διάγραμμα στο \autoref{fig:measurement_model} ο οριζόντιος άξονας αντιστοιχεί στη μέτρηση $z_t^k$ και ο κατακόρυφος στην πιθανότητα. 

\begin{figure}[!ht]
    \centering
    \includegraphics[width=0.8\textwidth]{./images/chapter4/measurement_model.png}
    \caption{Τα στοιχεία του μοντέλου για έναν αισθητήρα μέτρησης απόστασης}
    \label{fig:measurement_model}
\end{figure} 

Επομένως, το ζητούμενο μοντέλο $p(z_t | x_t, m)$ είναι ένα μείγμα των παρακάτω τεσσάρων πυκνοτήτων, καθεμία από τις οποίες αντιστοιχεί σε συγκεκριμένο τύπο σφάλματος.

\emph{Σωστή απόσταση με τοπικό θόρυβο μέτρησης:} Ακόμα και αν ο αισθητήρας μετράει σωστά την απόσταση από το πλησιέστερο αντικείμενο, η τιμή που επιστρέφει μπορεί να επηρεαστεί από κάποιο σφάλμα. Το τελευταίο οφείλεται στην περιορισμένη ανάλυση των αισθητήρων μέτρησης απόστασης, στις ατμοσφαιρικές επιδράσεις στο σήμα της μέτρησης κ.ο.κ. Αυτός ο θόρυβος μέτρησης μοντελοποιείται συνήθως με μια μικρού πλάτους κατανομή Gauss, όπως φαίνεται στο \autoref{fig:measurement_model}(a). Επομένως, η πιθανότητα μέτρησης δίνεται από τον παρακάτω τύπο:
\begin{equation*}
 p_{hit}(z_t^k | x_t, m) =
  \begin{cases}
    \eta N_(z_t^k, z_t^{k*}, \sigma_{hit}^2) & \quad \text{if } 0 < z_t^k < z_{max}\\
    0  & \quad \text{else}
  \end{cases}
\end{equation*}

\begin{equation*}
    N_(z_t^k, z_t^{k*}, \sigma_{hit}^2) = \frac{1}{\sqrt{2\pi\sigma_{hit}^2}} e^{-\frac{1}{2} \frac{(z_t^k - z_t^{k*})^2}{\sigma_{hit}^2} }
\end{equation*}

\emph{Μη αναμενόμενα αντικείμενα:} Τα περιβάλλοντα των κινητών ρομπότ είναι δυναμικά, ενώ οι χάρτες στατικοί. Αυτό έχει ως αποτέλεσμα τα αντικείμενα που δεν περιέχονται στο χάρτη να αποτελούν την αιτία για την οποία οι αισθητήρες μέτρησης απόστασης παράγουν αναπάντεχα μικρές αποστάσεις. Μια απλή μέθοδος για να αντιμετωπίσουμε τέτοιες καταστάσεις είναι να τις θεωρήσουμε ως θόρυβο. Σε τέτοιες περιπτώσεις η πιθανότητα των μετρήσεων απόστασης περιγράφεται από μια εκθετική κατανομή, όπως φαίνεται στο \autoref{fig:measurement_model}(b). Η παράμετρος $\lambda_{short}$ είναι μια εγγενής παράμετρος του μοντέλου μέτρησης. Η πιθανότητα $p_{short}(z_t^k | x_t, m)$ δίνεται από τον παρακάτω τύπο:
\begin{equation*}
 p_{short}(z_t^k | x_t, m) =
  \begin{cases}
    \eta \lambda_{short} e^{-\lambda_{short} z_t^k} & \quad \text{if } 0 < z_t^k < z_{max}\\
    0  & \quad \text{else}
  \end{cases}
\end{equation*}

\emph{Αποτυχίες:} Μερικές φορές τα εμπόδια δεν εντοπίζονται. Αυτό μπορεί να συμβεί σε αισθητήρες σόναρ εξαιτίας των κατευθυνόμενων ανακλάσεων ή στους αισθητήρες λέιζερ όταν ανιχνεύονται μαύρα αντικείμενα που απορροφούν το φως. Ένα ακόμη τυπικό παράδειγμα αποτυχίας είναι η μέτρηση μέγιστης απόστασης, ο αισθητήρας επιστρέφει την μέγιστη επιτρεπόμενη τιμή $z_{max}$. Η περίπτωση αυτή μοντελοποιείται με μια κατανομή σημειακής μάζας που είναι κεντραρισμένη στην τιμή $z_{max}$, όπως φαίνεται στο \autoref{fig:measurement_model}(c).
\begin{equation*}
 p_{max}(z_t^k | x_t, m) =
  \begin{cases}
    1 & \quad \text{if } z = z_{max}\\
    0  & \quad \text{else}
  \end{cases}
\end{equation*}

\emph{Τυχαίες μετρήσεις:}  Τέλος, οι αισθητήρες απόστασης παράγουν περιστασιακά εντελώς ανεξήγητες μετρήσεις. Για λόγους απλότητας μοντελοποιούμε αυτές τις μετρήσεις χρησιμοποιώντας μια ομοιόμορφη κατανομή που καλύπτει ολόκληρο το εύρος των μετρήσεων, όπως φαίνεται στο \autoref{fig:measurement_model}(d).
\begin{equation*}
 p_{rand}(z_t^k | x_t, m) =
  \begin{cases}
    \frac{1}{z_{max}} & \quad \text{if } 0 < z_t^k < z_{max}\\
    0  & \quad \text{else}
  \end{cases}
\end{equation*}

Οι συγκεκριμένες τέσσερις διαφορετικές κατανομές συνδυάζονται με τη χρήση ενός σταθμισμένου μέσου όρου, που ορίζεται από τις παραμέτρους $z_{hit}$, $z_{short}$, $z_{max}$ και $z_{rand}$ με το άθροισμα αυτών να ισούται με 1. Αυτό είναι εφικτό, καθώς θεωρούμε ότι οι μεμονωμένες μετρήσεις είναι υπό συνθήκη ανεξάρτητες.

\begin{algorithm}[!ht]
 \caption{Αλγόριθμος υπολογισμού πιθανότητας μιας σάρωσης απόστασης}
 \label{alg:range_model}
    \begin{algorithmic}[1]
        \State $q \gets 1$
        \For{k=1 to K} 
            \State {Υπολογισμός της $z_t^{k*}$ για την μέτρηση $z_t^k$ με ρίψη ακτίνων}
            \State $p \gets z_{hit} \cdot p_{hit}(z_t^k | x_t, m) + z_{short} \cdot p_{short}(z_t^k | x_t, m) + z_{max} \cdot  p_{max}(z_t^k | x_t, m) + z_{rand} \cdot p_{rand}(z_t^k | x_t, m)$
            \State $q \gets q \cdot p$
        \EndFor
        \State \Return $q$
    \end{algorithmic}
\end{algorithm}

Ο αριθμός των σωματιδίων παραμένει σταθερός σε όλη τη διάρκεια της εκτέλεσης του φίλτρου, με αποτέλεσμα μετά από κάποια βήματα το βάρος ορισμένων σωματιδίων να γίνεται πολύ μικρό και τα σωματίδια αυτά να μην έχουν καμία επίδραση στο φίλτρο. Για το λόγο αυτό, εφαρμόζεται μια διαδικασία, γνωστή ως \emph{Αναδειγματοληψία}, κάθε φορά που η διακύμανση των βαρών των σωματιδίων είναι μικρότερη από το μισό του αριθμού των συνολικών σωματιδίων. Η τιμή αυτή λέγεται $N_{eff}$ και περιγράφει το κατά πόσο όμοια είναι κατανεμημένα τα βάρη. Το βήμα της αναδειγματοληψίας είναι μια πιθανοτική υλοποίηση της ιδέας του Δαρβίνου για την επικράτηση του ισχυρότερου: επανασυγκεντρώνει το σύνολο των σωματιδίων σε περιοχές του χώρου καταστάσεων με μεγάλη εκ των υστέρων πιθανότητα. Με αυτόν τον τρόπο, εστιάζει τους υπολογιστικούς πόρους που χρησιμοποιούνται από τον αλγόριθμο του φίλτρου στις περιοχές του χώρου καταστάσεων που έχουν τη μεγαλύτερη σημασία.

\begin{algorithm}[!ht]
 \caption{Αναδειγματοληψία χαμηλής διακύμανσης}
 \label{alg:resampling}
    \begin{algorithmic}[1]
        \State $new\_particles \gets 0$
        \State $r \gets rand(0;M^{-1})$
        \State $c \gets w_t^{[1]}$
        \State $i \gets 1$
        \For{m=1 to M}
            \State $U \gets r + (m-1) \cdot M^{-1}$
            \While{U > c}
                \State $i \gets i + 1$
                \State $c = c + w_t^{[1]}$
            \EndWhile
            \State $new\_particles \gets new\_particles + w_t^{[i]}$
        \EndFor
        \State \Return $new\_particles$
    \end{algorithmic}
\end{algorithm}

Η στρατηγική που χρησιμοποιείται για την αναδειγματοληψία είναι αυτή της χαμηλής διακύμανσης και φαίνεται στον \autoref{alg:resampling}. Η μέθοδος αυτή, αντί να επιλέγει δείγματα ανεξάρτητα το ένα από το άλλο στη διαδικασία της αναδειγματοληψίας ακολουθεί μια στοχαστική διαδικασία. Υπολογίζεται ένας τυχαίος αριθμός και τα δείγματα επιλέγονται με βάση αυτόν, με πιθανότητα όμως που παραμένει ανάλογη του συντελεστή στάθμισης του δείγματος. Αυτό επιτυγχάνεται επιλέγοντας τον αριθμό $r$ στο διάστημα $[0, M^{-1}]$, όπου $M$ ο αριθμός των σωματιδίων που πρόκειται να ληφθούν. Έπειτα, επιλέγονται σωματίδια προσθέτοντας συνεχώς την σταθερή ποσότητα $M^{-1}$ στον αριθμό $r$ και επιλέγοντας το σωματίδιο που αντιστοιχεί στον αριθμό που προκύπτει. Με τον τρόπο αυτό, καλύπτεται ο χώρος με πιο συστηματικό τρόπο, ενώ επιτυγχάνεται δειγματοληψία με πολυπλοκότητα $\mathcal{O}(M)$, κάτι που είναι πολύ σημαντικό για ρομποτικές εφαρμογές που απαιτούν υψηλή απόδοση σε περιορισμένο υλικό.
\section{Πλήρης κάλυψη χώρου}
\label{section:coverage_impl}

Το πρόβλημα της κάλυψης χώρου έχει δύο βασικές εκδοχές: με χάρτη και χωρίς χάρτη. Στην εργασία αυτή, ο χάρτης είναι διαθέσιμος σε μορφή OctoMap και σκοπός είναι η πλοήγηση του ρομπότ μέχρι να καλυφθεί όλος ο χάρτης, ή τουλάχιστον όλα τα σημεία ενδιαφέροντος σε αυτόν. Υπάρχουν διάφοροι τρόποι για την επίλυση αυτού του προβλήματος, είτε οι αναλυτικοί, αλλά υπολογιστικά αργοί αλγόριθμοι, είτε οι ευριστικοί που με την χρήση ορισμένων πρακτικών κανόνων επιταχύνουν την αναζήτηση.

Βασική ιδέα είναι όλα τα αντικείμενα του χώρου να βρεθούν κάποια στιγμή στο οπτικό πεδίο ενός αισθητήρα που βρίσκεται πάνω στο ρομπότ. Στη συγκεκριμένη περίπτωση, πρόκειται για έναν αναγνώστη RFID με τα παρακάτω παραμετροποιήσιμα χαρακτηριστικά:
\begin{itemize}
    \item {Εμβέλεια}
    \item {Οριζόντιο και κάθετο πεδίο όρασης}
    \item {Τη μορφή του πεδίου κάλυψης (ορθογώνιο/κυκλικό)}
    \item {Το διάνυσμα που δίνει την κατεύθυνση του αναγνώστη}
\end{itemize}

Για να υπολογίσουμε τα σημεία στα οποία πρέπει να βρεθεί το ρομπότ, ώστε να πετύχει την πλήρη κάλυψη του χώρου, υποδειγματοληπτούμε τον χώρο με κάποιο σταθερό βήμα ως προς $x$ και $y$ και διαφορετικό βήμα ως προς $z$, φροντίζοντας πάντα να διατηρούνται οι αποστάσεις ασφαλείας του drone από οποιοδήποτε στατικό εμπόδιο στο περιβάλλον. Το βήμα δειγματοληψίας ως προς τον $z$ άξονα υπολογίζεται με βάση τη γωνία του κάθετου οπτικού πεδίου του αισθητήρα, με σκοπό κάθε σημείο του χώρου να βρεθεί εντός του πεδίου οράσεως τουλάχιστον δύο φορές. Συνεπώς, χρησιμοποιώντας τριγωνομετρία, υπολογίζεται με τον παρακάτω τύπο:
\begin{equation}
    z_{step} = range \cdot tan(fov_{vertical} / 2)
\end{equation}

\begin{figure}[!ht]
    \centering
    \includegraphics[width=0.8\textwidth]{./images/chapter4/z_step.png}
    \caption{Μετάβαση του drone από το ύψος Α στο ύψος Β}
    \label{fig:z_step}
\end{figure} 

Πιο συγκεκριμένα, στο \autoref{fig:z_step} το σημείο $P$ βρίσκεται εντός του οπτικού πεδίου του drone, όταν αυτό είναι στο ύψος $A$ και όταν είναι στο ύψος $B = A + range \cdot tan(fov_{vertical}/2)$. 

Στη συνέχεια, για κάθε ένα σημείο που προκύπτει από την δειγματοληψία ελέγχουμε εάν η θέση αυτή είναι ασφαλής για το ρομπότ. Οι έλεγχοι που πραγματοποιούνται είναι: αν η θέση αυτή ανήκει σε κατειλημμένο κόμβο του χάρτη μορφής OctoMap και εάν υπάρχουν εμπόδια σε απόσταση μικρότερη από την επιτρεπόμενη απόσταση ασφαλείας που έχει οριστεί. Με χρήση του προβαλλόμενου χάρτη σε δύο διαστάσεις, μπορούμε επιπλέον να ελέγξουμε αν το κελί της θέσης που μας ενδιαφέρει βρίσκεται σε χώρο που υπάρχει εμπόδιο και δεν έχει αναγνωριστεί με κανέναν από τους προηγούμενους τρόπους.

Εφόσον αυτοί οι έλεγχοι καταλήξουν στο συμπέρασμα ότι η θέση είναι ασφαλής για το drone, υπολογίζουμε τον προσανατολισμό ο οποίος προσδίδει καλύτερη ορατότητα του αντικειμένου. Για το λόγο αυτό, ελέγχουμε διαδοχικά διαφορετικές γωνίες και υπολογίζουμε το ποσοστό κάλυψης από την καθεμία. Ο αλγόριθμος που περιγράφει τον τρόπο υπολογισμού του ποσοστού κάλυψης για μία συγκεκριμένη κατεύθυνση φαίνεται στον \autoref{alg:calculate_coverage}. Αρχικά, με χρήση της συνάρτησης $castRay(\cdot)$ της βιβλιοθήκης OctoMap, υπολογίζουμε την θέση του σημείου που βρίσκεται στο κέντρο του οπτικού πεδίου του αισθητήρα. Στη συνέχεια, η θέση αυτή και η κατεύθυνση του αισθητήρα, δίνονται ως ορίσματα στην συνάρτηση υπολογισμού. Εκεί υπολογίζοντας το γινόμενο του κάθετου διανύσματος της επιφάνειας και του διανύσματος της κατεύθυνσης, προκύπτει μία μετρική αξιολόγησης της γωνίας αυτής. 

Για τον υπολογισμό του κάθετου διανύσματος της επιφάνειας, χρησιμοποιείται η συνάρτηση $getNormals(\cdot)$ που παρέχει επίσης η βιβλιοθήκη OctoMap. Η συνάρτηση αυτή επιστρέφει τα κάθετα διανύσματα των κορυφών των τριγώνων τα οποία περιέχουν το σημείο το οποίο εξετάζεται. Είναι πιθανό κάποιο σημείο να μην βρίσκεται σε κάποια κορυφή και να βρίσκεται στο εσωτερικό ενός τριγώνου. Στην περίπτωση αυτή, αναζητούμε το κάθετο διάνυσμα γειτονικών κόμβων σε συγκεκριμένη ακτίνα, καθώς θεωρούμε ότι βρίσκονται στην ίδια επιφάνεια και συνεπώς το κάθετο διάνυσμα θα ταυτίζεται. Τέλος, βρίσκουμε τον μέσο όρο των διανυσμάτων αυτών και το αποτέλεσμα θα χρησιμοποιηθεί για τον υπολογισμό της μετρικής.

\begin{algorithm}[!ht]
 \caption{Αλγόριθμος υπολογισμού καλύτερης γωνίας θέασης}
 \label{alg:calculate_coverage}
 \begin{algorithmic}[1]
    \Function{calculateCoverage}{point\_on\_wall, direction}
        \State $coverage \gets 0$
        \State $normals \gets point\_on\_wall.getNormals()$
        \While {$normals.size() = 0$}
            \State $normals \gets point\_on\_wall.getNormals()$ \Comment{Calculate normals in a larger bounding box around point\_on\_wall}
        \EndWhile
        \State $mean \gets normals.mean()$
        \State $coverage \gets direction \cdot mean$
        \State \Return $coverage$
    \EndFunction 
    \end{algorithmic}
\end{algorithm}

Μετά την εκτέλεση της διαδικασίας αυτής, αποθηκεύεται για κάθε θέση $(x, y, z)$ η τιμή του $yaw$ που προσφέρει την καλύτερη οπτική γωνία. Οι τιμές των roll και pitch θεωρούνται μηδενικές, καθώς είναι επιθυμητοί οι αργοί ελιγμοί, ώστε να επιτευχθεί ακρίβεια.

Στη συνέχεια, εφαρμόζεται μια περαιτέρω επεξεργασία σε αυτά, ώστε να αποκλειστούν σημεία τα οποία δεν είναι ορατά και επομένως δεν είναι προσβάσιμα από κανένα άλλο σημείο του χώρου. Για το λόγο αυτό, σε κάθε ένα από τα σημεία που έχουν προκύψει εφαρμόζεται ο αλγόριθμος πλησιέστερου γείτονα (\emph{Nearest Neighbor Algorithm}), ώστε να ελεγχθούν γειτονικά σημεία του καθενός και να αποκλειστούν αυτά τα οποία δεν είναι προσβάσιμα με άμεσο τρόπο. Πιο συγκεκριμένα, για κάθε σημείο ελέγχουμε να βρούμε γειτονικά σημεία που απέχουν απόσταση μικρότερη από το 0.75 του εύρους του αισθητήρα RFID και δεν έχουν θεωρηθεί ήδη ως προσβάσιμα από κάποιο άλλο σημείο. Ο έλεγχος της ορατότητας για κάθε γείτονα πραγματοποιείται με χρήση της συνάρτησης $computeRayKeys(\cdot)$ της βιβλιοθήκης OctoMap. Η διαδικασία αυτή εφαρμόζεται αναδρομικά για κάθε σημείο, μέχρι να μην υπάρχει άλλο διαθέσιμο σημείο. 

Για την οπτικοποίηση της επιφάνειας που καλύπτεται από τον αισθητήρα, δημιουργούμε έναν νέο χάρτη μορφής OctoMap, στον οποίο προστίθενται όλα τα σημεία που έχουν βρεθεί στον οπτικό πεδίο του. Έχοντας πλέον όλες τις θέσεις στις οποίες πρέπει να βρεθεί το ρομπότ με την σωστή κατεύθυνση, μπορούμε να υπολογίσουμε μια εκτίμηση του ποσοστού κάλυψης του χώρου σε \si{\cubic\metre}. Η διαδικασία αυτή περιγράφεται στον \autoref{alg:calculate_volume}.

\newcommand{\pluseq}{\mathrel{+}=}

\begin{algorithm}[!ht]
 \caption{Υπολογισμός όγκου ενός OctoMap}
 \label{alg:calculate_volume}
    \begin{algorithmic}[1]
        \State {$volume \gets 0$}
        \For{i = 1 to N}
            \If{ $node(i).z() > max\_height$ OR $node(i).z() < min\_height$}
                \State $continue$
            \EndIf
            \If{ $node(i)$ is Occupied}
                \State {$volume \pluseq node(i).size * node(i).size * node(i).size$}
            \EndIf
        \EndFor
        \State \Return $volume$
    \end{algorithmic}
\end{algorithm}

\begin{equation*}
    coverage (\%) = 100 * \frac{covered\_map_{volume}}{initial\_map_{volume}} 
\end{equation*}

Όπως φαίνεται στην γραμμή 3 του αλγορίθμου, περιορίζουμε τον χώρο σε ένα ελάχιστο και μέγιστο ύψος. Όσα σημεία βρίσκονται εκτός των ορίων αυτών δεν λαμβάνονται υπόψη στον υπολογισμό του όγκου. Ο λόγος που συμβαίνει αυτό είναι ο αποκλεισμός σημείων του χώρου, όπως το δάπεδο, που δεν παρουσιάζουν κανένα ενδιαφέρον για κάλυψη, αλλά και η εξάλειψη του θορύβου του χάρτη σε υψηλά σημεία. Επίσης, η ύπαρξη θορύβου στην αναπαράσταση του περιβάλλοντος δημιουργεί προβλήματα στην ακριβή αξιολόγηση της κάλυψης του χώρου. Συγκεκριμένα, υπάρχουν περιπτώσεις όπου δημιουργούνται σημεία τα οποία δεν είναι άμεσα ορατά από το ρομπότ και συνεπώς δεν θα έπρεπε να συνυπολογίζονται στον συνολικό όγκο. Αυτό συμβαίνει λόγω σφάλματος κατά την χαρτογράφηση του χώρου, όπου οι τοίχοι του περιβάλλοντος μπορούν να πάρουν τη μορφή που φαίνεται στο \autoref{fig:octomap_noise}.

\begin{figure}[!ht]
    \centering
    \includegraphics[width=0.50\textwidth]{./images/chapter4/octomap_noise.png}
    \caption{Σφάλμα του χάρτη στην αναπαράσταση επιφανειών}
    \label{fig:octomap_noise}
\end{figure}

Δηλαδή κάποια σημεία στις επιφάνειες αποτελούνται εσφαλμένα από δύο διαχοχικούς κόμβους και δεν προσφέρουν κάποια χρήσιμη πληροφορία σχετική με το χώρο. Για το λόγο αυτό, εφαρμόζουμε μια επιπλέον επεξεργασία του χάρτη προτού υπολογιστεί ο όγκος του. Η διαδικασία αυτή φαίνεται στο \autoref{fig:octomap_proprocess}, κατά την οποία αποκλείουμε τους εξωτερικούς κόμβους από τον υπολογισμό του όγκου, καθώς αρκεί η κάλυψη των άμεσα ορατών από τον αισθητήρα.

\begin{figure}[!ht]
    \centering
    \includegraphics[width=0.35\textwidth]{./images/chapter4/octomap_proprocess.png}
    \caption{Εύρεση σημείων που δεν λαμβάνονται υπόψη στον υπολογισμό του όγκου ενός OctoMap}
    \label{fig:octomap_proprocess}
\end{figure}

Με τα βήματα που περιγράφτηκαν έως τώρα γίνεται ένας τυχαίος υπολογισμός των σημείων που προσφέρουν την καλύτερη δυνατή κάλυψη του χώρου. Στη συνέχεια, πρέπει να βρεθεί η σειρά με την οποία το drone πρέπει να προσπελάσει αυτά τα σημεία, ώστε η κάλυψη του χώρου να γίνεται με όσο το δυνατόν πιο βέλτιστο τρόπο, είτε όσον αφορά το χρόνο κάλυψης, είτε την απόσταση που θα διανύσει. 

Για την επίλυση του προβλήματος αυτού, απαιτείται να το προσεγγίσουμε ως πρόβλημα του περιοδεύοντος πωλητή (\emph{Travelling Salesman Problem}). Εν συντομία, πρόκειται για την περίπτωση όπου ένας πωλητής πρέπει να επισκεφτεί $n$ διαφορετικές πόλεις και πρέπει να υπολογιστεί η συντομότερη και χαμηλότερου κόστους διαδρομή, ώστε να επισκεφτεί κάθε πόλη μόνο μία φορά. Πρόκειται για ένα NP-hard πρόβλημα, του οποίου όσο αυξάνεται ο αριθμός των πόλεων, τόσο αυξάνεται και η πολυπλοκότητά του. 

Αρχικά, επιχειρήθηκε η επίλυση του προβλήματος με τη χρήση του αλγορίθμου \emph{Hill climbing}. Όμως, το μεγάλο πλήθος των σημείων δεν επέτρεπε την εκτέλεση του αλγορίθμου σε λογικό χρόνο και η βελτίωση που υπήρχε ήταν πολύ μικρή. Όπως είναι λογικό, στην περίπτωση κάλυψης χώρου για μία βιομηχανική αποθήκη είναι σημαντική η πιθανότητα ο αριθμός των σημείων να είναι πολύ μεγάλος, με αποτέλεσμα να γίνεται αρκετά πολύπλοκη η επίλυση του προβλήματος με οποιονδήποτε τρόπο.

Για το λόγο αυτό, επεξεργαζόμαστε τα σημεία που έχουν προκύψει, ώστε να μειωθεί ο αριθμός τους και να λυθεί πιο εύκολα το πρόβλημα της ελαχιστοποίησης του κόστους της διαδρομής. Ως κόστος, θεωρούμε την απόσταση που θα διανύσει το drone για να επιτύχει την πλήρη κάλυψη.

Ένας τρόπος για να μειωθεί το πλήθος των σημείων είναι η μείωση των διαστάσεων αυτών. Τα σημεία έως τώρα είναι της μορφής $\kappa = (x, y, z, \theta)$. Το πλήθος αυτών θα μπορούσε να μειωθεί τόσες φορές, όσα είναι και τα διαφορετικά ύψη στα οποία θα πρέπει να βρεθεί το drone, κρατώντας μόνο την τετμημένη και τεταγμένη των σημείων. Ουσιαστικά, πρόκειται για μια προβολή των τρισδιάστατων σημείων στον δισδιάστατο χώρο. 

Πλέον, καθώς ο αριθμός των σημείων είναι αρκετά μικρότερος, μπορεί να εφαρμοστεί σε αυτά κάποιος αλγόριθμος για την εύρεση του βέλτιστου μονοπατιού. Δημιουργείται ένας μη κατευθυνόμενος γράφος με όλα τα σημεία $(x, y)$ που έχουν προκύψει, χρησιμοποιώντας την βιβλιοθήκη \emph{Boost Graph Library}\footnote{\href{https://www.boost.org/doc/libs/1\_66\_0/libs/graph/doc/index.html}{https://www.boost.org/doc/libs/1\_66\_0/libs/graph/doc/index.html}}. Για κάθε σημείο, προστίθεται ως γειτονικός κόμβος κάθε άλλο σημείο που απέχει απόσταση μικρότερη από το 0.75 του εύρους του αισθητήρα κάλυψης. Το βάρος μεταξύ των δύο κόμβων πρόκειται για την απόσταση αυτή.

Για την εύρεση του βέλτιστου μονοπατιού, χρησιμοποιούμε έναν συνδυασμό των αλγορίθμων Nearest Neighbor και Hill Climbing, ώστε να αξιοποιήσουμε γειτονικά σημεία και να προσπαθήσουμε να μειώσουμε τις αποστάσεις μεταξύ σημείων που βρίσκονται αρκετά μακριά με ευριστικό τρόπο. Για τον υπολογισμό της απόστασης μεταξύ των κόμβων στο γράφο, χρησιμοποιείται ο αλγόριθμος \emph{A*}. Πιο συγκεκριμένα, για κάθε ένα σημείο $(x, y)$ ελέγχουμε την απόσταση του από το επόμενο και το προηγούμενο του, καθώς ο τρόπος με τον οποίο δημιουργήθηκαν τα σημεία, επιτρέπει την ύπαρξη γειτονικών σημείων διαδοχικά στη δομή που αποθηκεύτηκαν. Επίσης, γνωρίζουμε την ελάχιστη απόσταση που μπορούν να έχουν δύο σημεία, η οποία είναι ίση με το βήμα δειγματοληψίας που χρησιμοποιήθηκε παραπάνω. Εάν κανένα από τα δύο γειτονικά δεν καλύπτουν τις απαιτήσεις απόστασης, επιχειρούμε να βρούμε το καλύτερο επόμενο σημείο επιλέγοντας έναν τυχαίο κόμβο και ελέγχοντας την απόσταση του και αν είναι άμεσα προσβάσιμος με εύκολο τρόπο (π.χ. ενώνονται με ευθεία γραμμή). Η αναζήτηση συνεχίζεται για πεπερασμένο αριθμό επαναλήψεων ή μέχρις ότου βρεθεί σημείο που να βρίσκεται σε ελάχιστη απόσταση. 

Η παραπάνω διαδικασία εκτελείται για συγκεκριμένο αριθμό επαναλήψεων, ώστε να βελτιωθεί όσο τον δυνατόν περισσότερο η λύση που προκύπτει. Συνολικά, η διαδικασία παρουσιάζεται στον \autoref{alg:calculate_optimal_path}.

\makeatletter
\newcommand{\HEADER}[1]{\State\underline{\textsc{#1}}}
    \newcommand{\ENDHEADER}{}
\makeatother
\newcommand{\STATEI}[1]{\State
    \begin{tabular}{@{}p{\dimexpr \textwidth-\labelwidth}@{}}%
        \hangindent \algorithmicindent
        \hangafter 1
        #1
    \end{tabular}
}

\begin{algorithm}[!ht]
 \caption{Αλγόριθμος εύρεσης βέλτιστου μονοπατιού}
 \label{alg:calculate_optimal_path}
 \begin{algorithmic}
    \Function{calculateOptimalPath}{points}
    \STATEI {Calculate initial total distance}
        \For {i in restarts}
            \STATEI {Starting from node 0}
            \While {there are nodes left}
                \HEADER{Check nearest neighbor}
                \STATEI {Check the node after and the node before}
                \If{at least one of them is visible and in a small distance}
                    \STATEI {Make this node the current node and skip Hill Climbing below}
                \EndIf
                \HEADER {Hill Climbing}
                \Do 
                     \STATEI {Find a random node and calculate its distance from current node}
                     \STATEI {Keep the node with the minimum distance}
                \doWhile{any distance found is larger than the minimum AND max number of iterations is not reached}
            \EndWhile
            \STATEI {Keep the order of points with the minimum total cost}
        \EndFor
    \EndFunction 
    \end{algorithmic}
\end{algorithm}

Εφόσον έχει υπολογιστεί ένα μονοπάτι που συνδέει όλα τα σημεία στον δισδιάστατο χώρο, η λύση μπορεί να μεταφερθεί ξανά στον τρισδιάστατο χώρο. Αυτό είναι δυνατόν να συμβεί με δύο τρόπους, με την οριζόντια ή κάθετη ένωση των σημείων. Στο \autoref{fig:revert} φαίνονται με πιο ξεκάθαρο τρόπο οι δύο αυτές μέθοδοι.

\begin{figure}[!ht]
    \centering
    \includegraphics[width=1.0\textwidth]{./images/chapter4/slice_lift.png}
    \caption{Οριζόντια και κάθετη ένωση των σημείων για επαναδιάταξη στον τρισδιάστατο χώρο}
    \label{fig:revert}
\end{figure} 

Τέλος, εφαρμόζεται ένα τελευταίο στάδιο επεξεργασίας του μονοπατιού. Ελέγχονται διαδοχικά όλα τα σημεία και αφαιρούνται αυτά τα οποία βρίσκονται στο ενδιάμεσο άλλων σημείων στους άξονες $x$, $y$ και $z$ και απέχουν μεταξύ τους την δεδομένη απόσταση δειγματοληψίας. Με τον τρόπο αυτό, επιδιώκουμε να έχουμε όσο το δυνατόν λιγότερα σημεία-στόχους, ώστε να είναι πιο ομαλή και συνεχόμενη η πορεία του ρομπότ. Πιο συγκεκριμένα, όπως φαίνεται στο \autoref{fig:postprocesspath}, τα σημεία 2, 5 και 8 παραλείπονται, καθώς είναι λογικό ότι κατά την μετάβαση του ρομπότ από το 1 στο 3, και αντίστοιχα για τα υπόλοιπα σημεία, θα διασχίσει το ενδιάμεσο αυτών, χωρίς να χρειάζεται να το ορίσουμε.

\begin{figure}[!h]
    \centering
    \includegraphics[width=0.65\textwidth]{./images/chapter4/postprocesspath.png}
    \caption{Τελευταίο στάδιο επεξεργασίας μονοπατιού κάλυψης}
    \label{fig:postprocesspath}
\end{figure} 

Πλέον, το σύνολο των σημείων που πρέπει να διασχίσει το ρομπότ, ώστε να πετύχει την πλήρη κάλυψη του χώρου είναι διαθέσιμο. Για την πλοήγηση του χρησιμοποιείται η βιβλιοθήκη OMPL, όπως περιγράφεται στο \autoref{section:ompl} και ο αλγόριθμος \emph{RRT*} για την δημιουργία ασφαλών μονοπατιών μεταξύ των σημείων (\autoref{fig:path_planning}). Για τον έλεγχο του drone, χρησιμοποιείται ο PID ελεγκτής θέσης, όπως περιγράφεται στο \autoref{section:pid}. Για την εξασφάλιση πιο ομαλής κίνησης, δεχόμαστε ότι το drone δεν χρειάζεται να βρεθεί ακριβώς στην θέση που του υποδηλώνουμε, αλλά είναι επιτρεπτή μια σχετικά μικρή απόσταση ώστε να θεωρηθεί ότι ο στόχος επετεύχθη και να αποσταλεί ο επόμενος. Έχει δημιουργηθεί επίσης ένας ROS κόμβος ο οποίος κατά την πλοήγηση του drone στο χώρο δείχνει σε πραγματικό χρόνο την περιοχή που καλύπτει ο αισθητήρας RFID και το ποσοστό κάλυψης κάθε στιγμή (\autoref{fig:coverage}). Αφού καλυφθεί όλος ο χώρος, όπως φαίνεται στο \autoref{fig:whole_coverage}, το drone επιστρέφει στην αρχική του θέση.

\begin{figure}[!ht]
    \centering
    \includegraphics[width=0.90\textwidth]{./images/chapter4/path_planning.png}
    \caption{Δημιουργία μονοπατιού στο χώρο με αποφυγή εμποδίων}
    \label{fig:path_planning}
\end{figure} 

\begin{figure}[!ht]
    \centering
    \includegraphics[width=0.90\textwidth]{./images/chapter4/coverage.png}
    \caption{Κάλυψη χώρου από το drone σε πραγματικό χρόνο}
    \label{fig:coverage}
\end{figure} 

\begin{figure}[!ht]
    \centering
    \includegraphics[width=0.90\textwidth]{./images/chapter4/whole_coverage.png}
    \caption{Ολοκληρωμένη κάλυψη χώρου από το drone}
    \label{fig:whole_coverage}
\end{figure} 


