\chapter{Θεωρητικό υπόβαθρο \& Εργαλεία}
\label{chapter:theory_tools}

Στο κεφάλαιο αυτό παρουσιάζονται βασικά θεωρητικά στοιχεία που κρίνονται απαραίτητα για την κατανόηση της παρούσας εργασίας. Συγκεκριμένα, παρουσιάζεται η αρχή λειτουργίας των μη επανδρωμένων οχημάτων, καθώς ο τρόπος που λειτουργεί ένας ελεγκτής PID. Επίσης, παρουσιάζονται τα εργαλεία και οι βιβλιοθήκες που χρησιμοποιήθηκαν για την υλοποίηση του συστήματος εντοπισμού θέσης και πλήρους κάλυψης χώρου από ένα μη επανδρωμένο αεροσκάφος. Ειδικότερα, θα αναλυθούν το μεσολειτουργικό σύστημα ROS, πάνω στο οποίο βασίστηκε όλη η υλοποίηση, το σύνολο ROS πακέτων Hector Quadrotor και οι βιβλιοθήκες OctoMap, Particle Filter και OMPL.


\section{Αρχή λειτουργίας των UAV}
\label{section:quadcopter}

Τα μη επανδρωμένα οχήματα χωρίζονται σε διαφορετικές κατηγορίες, ανάλογα με τον αριθμό και τη διάταξη των κινητήρων τους. Στην συγκεκριμένη εργασία, θα ασχοληθούμε με το τετρακόπτερο (Quadcopter), του οποίου η μορφή φαίνεται στο παρακάτω σχήμα.
\begin{figure}[!ht]
    \centering
    \includegraphics[scale=0.5]{./images/chapter3/QuadRotorPlus.png}
    \caption{Δομή ενός Quadcopter} 
    Πηγή: \href{https://dev.px4.io/en/airframes/airframe\_reference.html}{https://dev.px4.io/en/airframes/airframe\_reference.html}
    \label{fig:drone_1}
\end{figure}

Όπως βλέπουμε, διαθέτει τέσσερις έλικες τοποθετημένους σε ίσες αποστάσεις και συμμετρικά κατανεμημένους από το κέντρο μάζας του drone, οι οποίοι είναι υπεύθυνοι για όλες τις λειτουργίες, όπως την κάθετη απογείωση και την προσγείωση του οχήματος. Διαθέτει έξι βαθμούς ελευθερίας και τέσσερις ελεγχόμενες μεταβλητές, όσες και ο αριθμός των κινητήρων. Μέσω αυτών επιτυγχάνεται ο έλεγχος της θέσης και του ύψους στο οποίο βρίσκεται το Quadcopter. 

Οι πιο χαρακτηριστικές λειτουργίες είναι η ανύψωση από το έδαφος, η προσγείωση και η σταθερή αιώρηση (hover). Για την ανύψωση από το έδαφος απαιτείται οι έλικες να κινούνται με την ίδια ταχύτητα και με την κατεύθυνση που φαίνεται στο \autoref{fig:drone_1}, ενώ η προσγείωση επιτυγχάνεται με την μείωση της ταχύτητας των στροφέων. Η λειτουργία hover, δηλαδή η σταθεροποίηση σε κάποιο συγκεκριμένο ύψος, χρειάζεται την κίνηση των στροφέων με τέτοια ταχύτητα, ώστε η βαρυτική δύναμη να είναι ίση με τη δύναμη που ωθεί το drone προς τα πάνω. Όσον αφορά την περιστροφή του drone, οι έλικες πρέπει να κινούνται με αντίθετη φορά και διαφορετικές ταχύτητες ανά δύο. Συγκεκριμένα, στο \autoref{fig:drone_moves} για μία αριστερόστροφη περιστροφή οι έλικες 2 και 4 θα περιστραφούν πιο γρήγορα από τους έλικες 1 και 3. Για την κίνηση προς τα δεξιά, αυξάνουμε την περιστροφική ταχύτητα στους έλικες που βρίσκονται στην πλευρά προς την οποία θέλουμε να κινηθούμε.

\begin{figure}[!ht]
    \centering
    \includegraphics[width=0.9\textwidth]{./images/chapter3/drone_movements.png}
    \caption{Περιστροφή και ευθεία κίνηση ενός Quadcopter} 
    \label{fig:drone_moves}
\end{figure}

Οι κινήσεις οι οποίες μπορεί να πραγματοποιήσει το Quadcopter αναλύονται ως προς τους τρεις άξονες x,y και z με τα roll, pitch και yaw αντίστοιχα. Pitch είναι η κίνηση με την οποία το drone κινείται είτε μπροστά είτε πίσω, roll είναι η κίνηση δεξιά ή αριστερά, ενώ τέλος το yaw ορίζει την κατεύθυνση του drone.
\section{Proportional Integral Derivative (PID) ελεγκτής}
\label{section:pid}

Ο αναλογικός-ολοκληρωτικός-παραγωγικός (PID) ελεγκτής είναι ένας μηχανισμός ανατροφοδότησης βρόχων ελέγχου που χρησιμοποιείται ευρέως σε διάφορα συστήματα ελέγχου, όπως ο έλεγχος της θερμοκρασίας, της πίεσης και της ταχύτητας. Ο ελεγκτής αυτός προσπαθεί να εξαλείψει το σφάλμα ανάμεσα σε ένα επιθυμητό σημείο λειτουργίας και στην τρέχουσα τιμή μιας μεταβλητής. Η έξοδος του είναι μια διορθωτική δράση που ρυθμίζει την διαδικασία αναλόγως.

Η έξοδος του ελεγκτή PID εξαρτάται από τους τρεις όρους που τον αποτελούν, τον αναλογικό (proportional), τον ολοκληρωτικό (integral) και τον παραγωγικό (derivative). Κάθε ένας από αυτούς επηρεάζει διαφορετικά την έξοδο και ρυθμίζονται ανεξάρτητα μεταξύ τους. Πιο συγκεκριμένα, ο αναλογικός όρος αναλαμβάνει τη διόρθωση την εξόδου, αναλογικά με την τιμή της διαφοράς ανάμεσα στην επιθυμητή και την τρέχουσα τιμή. Ο ολοκληρωτικός όρος λαμβάνει υπόψη το άθροισμα των σφαλμάτων και ο παραγωγικός την παράγωγο του σφάλματος. Η επίδραση καθενός από αυτούς τους όρους καθορίζεται από μία σταθερά, η οποία πρέπει να ρυθμίζεται ξεχωριστά για κάθε πρόβλημα που χρησιμοποιείται ο ελεγκτής, καθώς δεν υπάρχει μία τιμή που να καλύπτει όλες τις περιπτώσεις.

Στο \autoref{fig:pid} φαίνεται αναλυτικά ο τρόπος λειτουργίας του ελεγκτή.

\begin{figure}[!ht]
    \centering
    \includegraphics[width=0.9\textwidth]{./images/chapter3/pid.png}
    \caption{Ο τρόπος λειτουργίας ενός PID ελεγκτή} 
    Πηγή: \href{https://se.mathworks.com/matlabcentral/mlc-downloads/downloads/submissions/58257/versions/2/screenshot.png}{https://se.mathworks.com/matlabcentral/mlc-downloads/downloads/submissions/58257/versions/2/screenshot.png}
    \label{fig:pid}
\end{figure}

Στη συγκεκριμένη εργασία, ο ελεγκτής PID χρησιμοποιείται για τον έλεγχο της θέσης του drone στο χώρο. Λαμβάνει ως είσοδο την θέση που βρίσκεται το drone και την επιθυμητή θέση και υπολογίζει την απόσταση τους. Η έξοδος του ελεγκτή υπολογίζεται με βάση τις τιμές των τριών όρων που αναλύθηκαν προηγουμένως και μετασχηματίζεται στη συνέχεια στην ταχύτητα που πρέπει να δοθεί στο drone, ώστε να φτάσει ομαλά στην επιθυμητή θέση. Χρησιμοποιούνται τέσσερις διαφορετικοί ελεγκτές, οι οποίοι υπολογίζουν ξεχωριστά το σφάλμα σε $x$, $y$, $z$ και $yaw$, καθώς είναι αδύνατο να βρεθεί μια διαμόρφωση που να καλύπτει όλες τις απαιτήσεις του ελεγκτή. Συνεπώς, ρυθμίζουμε ξεχωριστά την γραμμική ταχύτητα στον $x$, $y$ και $z$ άξονα και την περιστροφική ταχύτητα γύρω από τον $z$ άξονα, δηλαδή την κίνηση $yaw$. Για να εξασφαλίσουμε την πιο ομαλή μετάβαση του ρομπότ στην επιθυμητή θέση, εφαρμόζουμε έναν διαδοχικό έλεγχο των τεσσάρων μεταβλητών. Πιο συγκεκριμένα, πρώτα ρυθμίζεται το ύψος και η θέση $(x,y)$ του drone και τέλος η κατεύθυνση του.

Για τον έλεγχο της θέσης, οι τρεις όροι επηρεάζουν με τους παρακάτω τρόπους την έξοδο του ελεγκτή και συνεπώς την θέση του drone: 
\begin{itemize}
    \item {Αναλογικός: Όσο πιο μεγάλη είναι η απόσταση του drone από την επιθυμητή θέση, τόσο μεγαλύτερη είναι και η επίδραση του όρου αυτού. Αντίθετα, αν πλησιάζει στο στόχο η επίδραση μειώνεται.}
    \item {Ολοκληρωτικός: Εάν το σφάλμα είναι μικρό για μεγάλο χρονικό διάστημα ή υψηλό για σύντομο χρονικό διάστημα αυξάνεται η επίδραση αυτού του όρου. Αντίθετα, αν το σφάλμα έχει μία μέτρια τιμή, περιορίζεται η επίδραση του. Ο όρος αυτός είναι πολύ σημαντικός για τα UAVs σε περιβάλλοντα με υψηλό αέρα, καθώς διατηρεί σταθερή τη θέση του από τις παρεμβολές. Καθώς η εργασία αυτή πραγματοποιείται εξ' ολοκλήρου σε περιβάλλον προσομοίωσης και δεν υπάρχουν τέτοιου είδους επιδράσεις, ο όρος αυτός αγνοείται.}
    \item {Παραγωγικός: Όσο αυξάνεται το σφάλμα, αυξάνεται και η επίδραση του όρου, ενώ όσο μειώνεται το σφάλμα παράγει αρνητική έξοδο, ώστε να μειωθεί η περίπτωση υπερύψωσης από τον αναλογικό όρο. Σε περίπτωση που δεν μεταβάλλεται το σφάλμα, ο όρος αυτός δεν επιδρά καθόλου. Μεγάλες τιμές του όρου αυτού μπορούν να οδηγήσουν σε αστάθεια και δονήσεις του drone, λόγω των απότομων μεταβολών.}
\end{itemize}

\section{Robot Operating System (ROS)}
\label{section:ROS}

Το Robot Operating System \cite{ros2009} είναι το πιο διαδεδομένο μεσολειτουργικό σύστημα για ρομποτικές εφαρμογές. Αποτελείται από ένα σύνολο βιβλιοθηκών και εργαλείων τα οποία στοχεύουν στην απλοποίηση της δημιουργίας πολύπλοκων και αξιόπιστων εφαρμογών που περιλαμβάνουν τη χρήση ρομποτικών πρακτόρων. Το ROS είναι ανοικτού κώδικα και έχει διαμορφωθεί με την συνεισφορά πολλών εθελοντών ανά τον κόσμο. Ο λόγος που δημιουργήθηκε είναι το γεγονός ότι η δημιουργία λογισμικού για ρομποτ είναι αρκετά δύσκολη, ιδιαίτερα όσο αυξάνεται το εύρος των ρομποτικών εφαρμογών. Διαφορετικά ρομπότ διαθέτουν και διαφορετικό υλικό (hardware) και έχουν διαφορετικές λειτουργικές απαιτήσεις ανάλογα με το σκοπό χρήσης τους. Συνεπώς, η δημιουργία ενός κοινού πλαισίου ανάπτυξης λογισμικού για ρομπότ, οδήγησε στην απλοποίηση αυτής της διαδικασίας με την ύπαρξη έτοιμων τμημάτων κώδικα που εκτελούν βασικές λειτουργίες σε κάθε εφαρμογή και την επαναχρησιμοποίησή τους.

Το ROS παρέχει όσα θα περίμενε κανείς και από ένα λειτουργικό σύστημα, δηλαδή αφαίρεση υλικού (hardware), έλεγχο συσκευών σε χαμηλό επίπεδο, ανταλλαγή μηνυμάτων μεταξύ των διεργασιών και διαχείριση πακέτων. Η δομή του βασίζεται στην αρχιτεκτονική peer-to-peer, όπου η κάθε διαδικασία είναι ένας κόμβος (node) \footnote{\href{http://wiki.ros.org/Nodes}{http://wiki.ros.org/Nodes}} στον γράφο. Οι κόμβοι μπορούν να είναι κατανεμημένοι σε διαφορετικά συστήματα, έχουν όμως τη δυνατότητα να επικοινωνούν μέσω της δομής επικοινωνίας του ROS. Υπάρχουν υλοποιημένες διάφορες μορφές επικοινωνίας, όπως η σύγχρονη επικοινωνία μέσω των services\footnote{\href{http://wiki.ros.org/Services}{http://wiki.ros.org/Services}}, η ασύγχρονη ροή δεδομένων σε topic\footnote{\href{http://wiki.ros.org/Topics}{http://wiki.ros.org/Topics}} και η αποθήκευση δεδομένων στον Parameter Server\footnote{\href{http://wiki.ros.org/Parameter\%20Server}{http://wiki.ros.org/Parameter\%20Server}}. Παρόλο που το ROS δεν είναι ένα σύστημα πραγματικού χρόνου, μπορεί να χρησιμοποιηθεί με κώδικα πραγματικού χρόνου, γεγόνός που το καθιστά αξιόπιστο για λειτουργία σε ρομποτικά συστήματα που απαιτούν άμεση απόκριση.

Ενδεικτικά, η αρχιτεκτονική του ROS φαίνεται στο \autoref{fig:ros_arch}. Ο κόμβος ROS Master δημιουργεί και συντηρεί τον αρχιτεκτονικό γράφο του συστήματος. Επιτρέπει σε κάθε ξεχωριστό ROS κόμβο να εντοπίσει κάποιον άλλον με τον οποίο θέλει να έρθει σε επικοινωνία.

\begin{figure}[!ht]
    \centering
    \includegraphics[width=1.0\textwidth]{./images/chapter3/ros101-2.png}
    \caption{Η αρχιτεκτονική του ROS} 
    Πηγή: \href{https://robohub.org/ros-101-intro-to-the-robot-operating-system/}{https://robohub.org/ros-101-intro-to-the-robot-operating-system/}
    \label{fig:ros_arch}
\end{figure}

% maybe add topics and messages we used and TF ?? 
\section{Hector Quadrotor}
\label{section:hector}

Το Hector Quadrotor\footnote{\href{http://wiki.ros.org/hector_quadrotor}{http://wiki.ros.org/hector\_quadrotor}} stack \cite{2012simpar_meyer} είναι ένα σύνολο ROS πακέτων που αφορούν την μοντελοποίηση, τον έλεγχο και την προσομοίωση σε Quadcopter. Η πρώτη του έκδοση δημοσιεύτηκε το 2012 από την ομάδα Team Hector του Πολυτεχνείου του Darmstadt.

Χρησιμοποιείται το περιβάλλον προσομοίωσης Gazebo\footnote{\href{http://gazebosim.org/}{http://gazebosim.org/}}, καθώς περιλαμβάνει όλους τους φυσικούς νόμους που υπάρχουν σε ένα πραγματικό περιβάλλον και ένα μεγάλο σύνολο ρομπότ, περιβάλλοντων και αισθητήρων που μπορούν να χρησιμοποιηθούν. Για την περιγραφή των χαρακτηριστικών του drone στον προσομοιωτή, χρησιμοποιείται ένα μοντέλο URDF\footnote{\href{http://wiki.ros.org/urdf/XML/model}{http://wiki.ros.org/urdf/XML/model}} που περιλαμβάνει σημαντικά χαρακτηριστικά, όπως η μάζα και η αδράνεια των τμημάτων του quadcopter. Επίσης μπορούν να προστεθούν ή και να μεταβληθούν με ευκολία οι αισθητήρες του.

Το συγκεκριμένο μοντέλο διαθέτει IMU, βαρομετρικό αισθητήρα για την προσομοίωση της στατικής πίεσης σε συγκεκριμένα υψομέτρα, αισθητήρα απόστασης sonar για την εκτίμηση του ύψους, αισθητήρα μαγνητικού πεδίου για τον υπολογισμό της διεύθυνσης του drone και τέλος δέκτη GPS. Εκτός αυτών, υπάρχει η επιλογή για την χρήση κάμερας και laser, ανάλογα με τις ανάγκες του περιβάλλοντος και της εφαρμογής.

\begin{figure}[!ht]
    \centering
    \includegraphics[width=0.8\textwidth]{./images/chapter3/hector.png}
    \caption{Το μοντέλο του Hector Quadrotor στον προσομοιωτή Gazebo} 
    \label{fig:hector}
\end{figure}

Το πακέτο αυτό χρησιμοποιήθηκε για την μοντελοποίηση του drone στην συγκεκριμένη διπλωματική εργασία, καθώς περιλαμβάνει όλα τα απαραίτητα κομμάτια κώδικα για την ρεαλιστική προσομοίωση του συστήματος. Συνεπώς, με την χρήση και την τροποποίηση αυτών, δόθηκε έμφαση στην επίλυση του προβλήματος και στη δημιουργία μιας λύσης που θα μπορεί να χρησιμοποιηθεί με οποιοδήποτε μοντέλο drone. 

Πιο συγκεκριμένα, μεταβλήθηκαν οι αισθητήρες του drone, ώστε να ταιριάζουν με τις απαιτήσεις του προβλήματος, ενώ χρησιμοποιήθηκαν αυτούσιοι οι αεροδυναμικοί παράμετροι και οι ελεγκτές που μεταφράζουν την ταχύτητα που δίνεται στους άξονες σε ομαλή κίνηση του αεροσκάφους.
\section{Octomap}
\label{section:octomap}

Η βιβλιοθήκη OctoMap \cite{hornung13auro} υλοποιεί μία αναπαράσταση του τρισδιάσταστου χώρου, παρέχοντας τις απαραίτητες δομές δεδομένων και αλγορίθμους χαρτογράφησης σε C++, με τρόπο κατάλληλο για χρήση σε  ρομποτικές εφαρμογές. Η υλοποίηση του χάρτη είναι σε μορφή οκταδικού δέντρου (octree), το οποίο προσφέρει σημαντικά πλεονεκτήματα. Είναι δυνατή η απεικόνιση τόσο του ελεύθερου όσο και του άγνωστου χώρου, ενώ παράλληλα υποστηρίζεται η προσθήκη νέων πληροφοριών οποιαδήποτε στιγμή στο χάρτη. Στο \autoref{fig:octree} φαίνεται η αποθήκευση ελεύθερων και κατειλημμένων κελιών στο οκταδικό δέντρο. Η ανανέωση γίνεται με πιθανοτικό τρόπο, καθώς λαμβάνει υπόψη πιθανό θόρυβο των αισθητήρων και δυναμικές μεταβολές του περιβάλλοντος. Επίσης, το μέγεθος του χάρτη δεν χρειάζεται να είναι γνωστό εκ των προτέρων. Είναι δυνατή η δυναμική επέκταση του και η προβολή του με διαφορετική ανάλυση, ανάλογα με τις ανάγκες του προβλήματος. Τέλος, η αποθήκευση του σε μορφή octree, οδηγεί στην αποδοτική αποθήκευση του τόσο στη μνήμη, όσο και στο δίσκο.

\begin{figure}[!ht]
    \centering
    \includegraphics[width=0.8\textwidth]{./images/chapter3/octree.png}
    \caption{Ογκομετρική απεικόνιση σε octree και το αντίστοιχο οκταδικό δέντρο} 
    \label{fig:octree}
\end{figure}


\section{Βιβλιοθήκη Particle Filter}
\label{section:libpf}

Η βιβλιοθήκη που υλοποιεί το φίλτρο σωματιδίων για τον αλγόριθμο εκτίμησης θέσης είναι η libPF\footnote{\href{https://github.com/stwirth/libPF}{https://github.com/stwirth/libPF}}. Το φίλτρο σωματιδίων είναι μια συλλογή καταστάσεων (States) τις οποίες παρακολουθεί, τις αξιολογεί με βάση μία τιμή βάρους που προκύπτει από το μοντέλο αξιολόγησης και θεωρεί την κατάσταση με το μεγαλύτερο βάρος ως την πραγματική. Έπειτα, πραγματοποιείται μια δειγματοληψία των σωματιδίων, ανάλογα με την κατάσταση των βαρών και αξιολογούνται τα επόμενα σωματίδια.

Για την λειτουργία του φίλτρου απαιτείται ο ορισμός την κατάστασης που παρακολουθείται, ενός μοντέλου κίνησης και ενός μοντέλου αξιολόγησης για τις καταστάσεις. Εκτός αυτών, μπορεί να οριστεί και ένα μοντέλο κατανομής καταστάσεων, για την περίπτωση που η αρχική θέση είναι γνωστή και παρέχεται στο φίλτρο.

Όσον αφορά τη διαδικασία που ακολουθεί ένα Particle Filter, αρχικά δημιουργείται το φίλτρο και προσδιορίζονται οι παράμετροι του μη γραμμικού συστήματος. Εάν θέλουμε να λύσουμε το πρόβλημα της καθολικής εύρεσης θέσης (Global Localization) τα σωματίδια αρχικοποιούνται σε τυχαίες θέσεις μέσα στο χώρο, ενώ αν γνωρίζουμε την αρχική θέση τα σωματίδια αρχικοποιούνται με βάση μια κανονική κατανομή γύρω από αυτήν. Το αρχικό βάρος όλων των σωματιδίων είναι ίδιο. Στην συνέχεια, τα σωματίδια κινούνται σύμφωνα με το κινηματικό μοντέλο του ρομποτικού πράκτορα, όπως αυτό λαμβάνεται από την οδομετρία ή άλλους σχετικούς αισθητήρες. Επομένως, τα σωματίδια βρίσκονται σε μια καινούρια κατάσταση και πρέπει να αξιολογηθούν, ώστε να εκτιμηθεί η νέα τιμή του βάρους τους. Η αξιολόγηση γίνεται σύμφωνα με τις μετρήσεις του αισθητήρα απόστασης που βρίσκεται στο drone. Η επαναδειγματοληψία των σωματιδίων γίνεται όταν ο αριθμός των αποτελεσματικών σωματιδίων ($N_{eff}$) είναι μικρότερος από το μισό του συνολικού αριθμού των σωματιδίων. Η τιμή του $N_{eff}$ δίνεται από τον παρακάτω τύπο:
\begin{equation*}
     N_{eff} = \frac{1}{\sum_{i=1}^{N_s} (w_k^i)^2}
\end{equation*}

Η βιβλιοθήκη libPF επιτρέπει επίσης την εκτέλεση της επαναδειγματοληψίας είτε σε κάθε βήμα εκτέλεσης του φίλτρου, μια δυνατότητα που επιβαρύνει υπολογιστικά το σύστημα, ή να μην συμβαίνει ποτέ, με αποτέλεσμα να υπάρχει η πιθανότητα κάποια σωματίδια να μηδενίσουν το βάρος τους μετά από κάποια βήματα εκτέλεσης του φίλτρου και να χαθεί η επίδρασή τους.
\section{The Open Motion Planning Library (OMPL)}
\label{section:ompl}

Η βιβλιοθήκη OMPL\footnote{\href{https://ompl.kavrakilab.org/}{https://ompl.kavrakilab.org/}} περιέχει πληθώρα αλγορίθμων που σχετίζονται με το πρόβλημα της πλοήγησης στις ρομποτικές εφαρμογές \cite{6377468}. Οι αλγόριθμοι που περιέχει όπως τα Rapidly-expanding Random Trees, Probabilistic Roadmap Method και άλλοι, βασίζονται σε δειγματοληπτικές μεθόδους και περιέχουν επιλύσεις για χώρο κάθε διάστασης. Είναι ανοικτού κώδικα, γραμμένη σε C++ και πλήρως συμβατή με το ROS, ενώ έχει χρησιμοποιηθεί από την επιστημονική κοινότητα τόσο για εκπαιδευτικούς όσο και για ερευνητικούς σκοπούς.

\begin{figure}[!ht]
    \centering
    \includegraphics[width=0.8\textwidth]{./images/chapter3/gui_path.png}
    \caption{Σχεδιασμός μονοπατιού με την OMPL} 
    \label{fig:ompl}
\end{figure}

Η βασική ιδέα στην οποία βασίζονται οι αλγόριθμοι αυτοί είναι ο διαχωρισμός του χώρου σε σημεία καταστάσεων και η ένωση των σημείων αυτών ως κορυφές σε έναν γράφο. Οι ακμές του γράφου υποδηλώνουν όλα τα εφικτά μονοπάτια μέσα στον χώρο. Η μορφή των σημείων εξαρτάται από το αντικείμενο για το οποίο εφαρμόζεται το path planning και τις δυνατές κινήσεις του, δηλαδή για ένα επίγειο ρομπότ αποτελείται από την κίνηση ως προς τους άξονες $x$ και $y$, ενώ για ένα drone θα είχε επιπλέον την κίνηση στον άξονα $z$, δηλαδή το ύψος στο οποίο βρίσκεται. Επίσης, περιέχει έλεγχο των καταστάσεων, ώστε να αποκλειστούν θέσεις οι οποίες είναι πιθανό να επιφέρουν σύγκρουση ή βρίσκονται εκτός των ορίων του περιβάλλοντος.

Η συγκεκριμένη βιβλιοθήκη προτιμήθηκε καθώς μπορεί να δεχτεί ως είσοδο έναν χάρτη σε μορφή OctoMap και να πραγματοποιήσει σχεδιασμό διαδρομής μέσα σε αυτόν. Επίσης, μετά τον σχεδιασμό των σημείων που απαιτούνται για την επίτευξη του στόχου, πραγματοποιείται μια εξομάλυνση του μονοπατιού, ανάλογα με την μετρική που δείχνει το πόσο ομαλό είναι το μονοπάτι. Με τον τρόπο αυτό, εξασφαλίζουμε την πλοήγηση με αποφυγή των γνωστών στο χώρο εμποδίων. Στο \autoref{fig:ompl} βλέπουμε ενδεικτικά την χρήση της βιβλιοθήκης για την δημιουργία ενός μονοπατιού στο χώρο.

Ο αλγόριθμος που χρησιμοποιήθηκε για την δημιουργία των μονοπατιών είναι ο RRT* (Optimal RRT) \cite{DBLP:journals/corr/abs-1105-1186}. Πρόκειται για μια βελτιωμένη εκδοχή του αλγορίθμου RRT \cite{lavalle1999}, ο οποίος εγγυάται την σύγκλιση σε βέλτιστη λύση, ενώ ο χρόνος εκτέλεσης του είναι σταθερός παράγοντας του χρόνου εκτέλεσης του RRT. 