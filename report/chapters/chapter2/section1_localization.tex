\section{Εντοπισμός Θέσης}
\label{section:localization}

Η εύρεση της θέσης ενός ρομπότ σε εσωτερικό περιβάλλον παρουσιάζει σημαντικές δυσκολίες. Η έλλειψη του αισθητήρα GPS οδηγεί στην ανάγκη για εύρεση άλλων αξιόπιστων μεθόδων υπολογισμού της θέσης και του προσανατολισμού ενός UAV. Αρκετά συχνά συναντάμε στην βιβλιογραφία προσεγγίσεις που βασίζονται κατά κύριο λόγο σε μεθόδους υπολογιστικής όρασης. Αυτό συμβαίνει, καθώς το κόστος, το μικρό βάρος αλλά και η πληροφορία που μπορεί να παρέχει μια RGB-D κάμερα διευκολύνουν την επίλυση του προβλήματος. Το βασικό μειονέκτημα των μεθόδων αυτών είναι η πολυπλοκότητα της επεξεργασίας και οι υψηλές υπολογιστικές απαιτήσεις για επεξεργασία των δεδομένων σε πραγματικό χρόνο. Το πρόβλημα του εντοπισμού θέσης μπορεί να χωριστεί σε δύο διαφορετικές κατηγορίες, ανάλογα με το εάν είναι γνωστή η αρχική θέση του ρομπότ μέσα στο χώρο ή όχι. 

Οι Perez-Grau κ.α. \cite{francisco2017perez} προτείνουν την χρήση του κλασικού αλγορίθμου Monte Carlo Localization \cite{amcl}, ο οποίος χρησιμοποιεί πιθανοτικές μεθόδους και φίλτρο σωματιδίων για την εύρεση της θέσης, σε συνδυασμό με μία κάμερα RGB-D και ορισμένους ραδιοπομπούς τοποθετημένους σε γνωστά σημεία μέσα στο χάρτη. Η απαιτούμενη οδομετρία για τα σωματίδια του AMCL προέρχεται από οπτική οδομετρία (visual odometry), ενώ ο υπολογισμός της απόστασης από τους ραδιοπομπούς βοηθάει στην εξάλειψη του συσσωρευμένου σφάλματος που προκύπτει. Επίσης, κάθε λήψη του αισθητήρα βάθους δημιουργεί μια τρισδιάστατη αναπαράσταση του χώρου. Αυτή συγκρίνεται με τον ήδη υπάρχοντα χάρτη και εάν βρεθούν κοινά σημεία, υπολογίζεται μία εκτίμηση της θέσης του ρομπότ μέσα στο γνωστό περιβάλλον και η πληροφορία αυτή ενισχύει την εκτίμηση του AMCL.

Μία άλλη προσέγγιση είναι η χρήση πολλαπλών RGB καμερών, όπως παρουσιάζεται στο \cite{6224750}. Παρόλο που δεν αναφέρεται σε UAV, το αποτέλεσμα είναι η εκτίμηση θέσης με 6 βαθμούς ελευθερίας, που σημαίνει ότι μπορεί να χρησιμοποιηθεί και για drones. Συγκεκριμένα, χρησιμοποιείται μια προ-επεξεργασμένη αναπαράσταση του χώρου σε μορφή τρισδιάστατων σημείων (PointCloud) και μια συνάρτηση κόστους, ώστε να υπολογιστεί η θέση με βάση την απόσταση των διαδοχικών λήψεων και την απόκλιση των λήψεων από τον γνωστό χάρτη.

Οι Ok, Greene και Roy \cite{7487651} χρησιμοποιούν μία RGB-D κάμερα, μόνο για την δημιουργία του χάρτη, ενώ η διαδικασία του εντοπισμού θέσης βασίζεται αποκλειστικά σε RGB εικόνα. Συγκεκριμένα, δημιουργούν εικονικές λήψεις του περιβάλλοντος από διάφορα σημεία κατανεμημένα στο χάρτη και στη συνέχεια κάθε εικόνα που λαμβάνεται από την κάμερα συγκρίνεται με τις αρχικές λήψεις, χρησιμοποιώντας οπτικά χαρακτηριστικά. Είναι σημαντικό να αναφερθεί ότι η υλοποίηση αυτή δεν απαιτεί μεγάλη υπολογιστική ισχύ, λόγω της μείωσης του ρυθμού λήψης εικόνων.

Οι Beul κ.α. \cite{Beul2017} αναφέρονται στην αυτόνομη πλοήγηση των UAVs σε μεγάλες αποθήκες. Οι αισθητήρες που χρησιμοποιούνται είναι ένα οριζόντιο και ένα κάθετο lidar, γωνίας 270° το κάθε ένα, τρία ζευγάρια από στερεοσκοπικές κάμερες, ένα IMU και ένας αναγνώστης RFID. Με χρήση της οπτικής οδομετρίας, υπολογίζεται η κίνηση του drone, ενώ οι αποστάσεις που προκύπτουν από τους αισθητήρες laser, μετατρέπονται σε μορφή PointCloud και συγκρίνονται οι επιφάνειες του με τις επιφάνειες του υπάρχοντα χάρτη. Όσον αφορά την πλοήγηση του drone στο χώρο, χρησιμοποιείται ο αλγόριθμος A* για την εύρεση μονοπατιών σε έναν διακριτοποιημένο κόσμο σε μορφή OctoMap, ενώ παράλληλα χρησιμοποιείται και τοπική αποφυγή εμποδίων για την περίπτωση ύπαρξης δυναμικού περιβάλλοντος.

Μία ακόμη εφαρμογή με UAVs σε χώρο αποθήκης παρουσιάζεται στο \cite{8392775}. Στην περίπτωση αυτή, χρησιμοποιούνται έναν αισθητήρας απόστασης lidar γωνίας 360° και εύρους 100 μέτρων, δύο κάμερες, ένα IMU και ένας αναγνώστης RFID. Η αποτελεσματικότητα της μεθόδου φαίνεται από το γεγονός ότι το drone είχε τη δυνατότητα να κινείται με σχετικά μεγάλη ταχύτητα (2.1 μέτρα/δευτερόλεπτο) χωρίς αυτό να επηρεάζει την αξιοπιστία της εκτίμησης θέσης και της πλοήγησης. Με την δημιουργία τρισδιάστατων προσωρινών χαρτών στους οποίους προστίθενται συνεχώς τα σημεία που προκύπτουν από τον αισθητήρα απόστασης, διευκολύνεται η εύρεση της θέσης με βάση τον αρχικό χάρτη.

Σε όλες τις προηγούμενες περιπτώσεις, το drone μπορούσε να διατηρεί μία απόσταση ασφαλείας από τα αντικείμενα του χώρου. Αντίθετα, στο \cite{Fang-2017-5588} χρησιμοποιείται ένα μη επανδρωμένο αεροσκάφος για τον έλεγχο και την εκτίμηση καταστροφών μέσα σε ένα περιβάλλον πλοίου με χαμηλή ορατότητα. Το περιβάλλον αυτό περιέχει στενά περάσματα, πόρτες και μικρά αντικείμενα, συνθήκες οι οποίες δεν επιτρέπουν την χρήση των μεθόδων που παρουσιάστηκαν στις προηγούμενες βιβλιογραφικές αναφορές. Για την πλοήγηση συνδυάζεται η οπτική οδομετρία που προέρχεται από μια RGB-D κάμερα με τις μετρήσεις ενός IMU, χρησιμοποιώντας ένα φίλτρο Kalman, ώστε να υπολογιστεί η ταχύτητα. Επίσης, για την βελτίωση της εκτίμησης θέσης χρησιμοποιείται ένας αλγόριθμο που χρησιμοποιεί Φίλτρα Σωματιδίων (Particle Filters) και τροφοδοτείται από την οδομετρία. Παράλληλα, χρησιμοποιείται μια υπέρυθρη κάμερα, ώστε να μην επηρεάζεται από την χαμηλή ορατότητα του περιβάλλοντος και να ανιχνεύει τις περιοχές υψηλού κινδύνου λόγω πυρκαγιάς.
