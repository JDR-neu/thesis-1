\section{Proportional Integral Derivative (PID) ελεγκτής}
\label{section:pid}

Ο αναλογικός-ολοκληρωτικός-παραγωγικός (PID) ελεγκτής είναι ένας μηχανισμός ανατροφοδότησης βρόχων ελέγχου που χρησιμοποιείται ευρέως σε διάφορα συστήματα ελέγχου, όπως ο έλεγχος της θερμοκρασίας, της πίεσης και της ταχύτητας. Ο ελεγκτής αυτός προσπαθεί να εξαλείψει το σφάλμα ανάμεσα σε ένα επιθυμητό σημείο λειτουργίας και στην τρέχουσα τιμή μιας μεταβλητής. Η έξοδος του είναι μια διορθωτική δράση που ρυθμίζει την διαδικασία αναλόγως.

Η έξοδος του ελεγκτή PID εξαρτάται από τους τρεις όρους που τον αποτελούν, τον αναλογικό (proportional), τον ολοκληρωτικό (integral) και τον παραγωγικό (derivative). Κάθε ένας από αυτούς επηρεάζει διαφορετικά την έξοδο και ρυθμίζονται ανεξάρτητα μεταξύ τους. Πιο συγκεκριμένα, ο αναλογικός όρος αναλαμβάνει τη διόρθωση την εξόδου, αναλογικά με την τιμή της διαφοράς ανάμεσα στην επιθυμητή και την τρέχουσα τιμή. Ο ολοκληρωτικός όρος λαμβάνει υπόψη το άθροισμα των σφαλμάτων και ο παραγωγικός την παράγωγο του σφάλματος. Η επίδραση καθενός από αυτούς τους όρους καθορίζεται από μία σταθερά, η οποία πρέπει να ρυθμίζεται ξεχωριστά για κάθε πρόβλημα που χρησιμοποιείται ο ελεγκτής, καθώς δεν υπάρχει μία τιμή που να καλύπτει όλες τις περιπτώσεις.

Στο \autoref{fig:pid} φαίνεται αναλυτικά ο τρόπος λειτουργίας του ελεγκτή.

\begin{figure}[!ht]
    \centering
    \includegraphics[width=0.9\textwidth]{./images/chapter3/pid.png}
    \caption{Ο τρόπος λειτουργίας ενός PID ελεγκτή} 
    Πηγή: \href{https://se.mathworks.com/matlabcentral/mlc-downloads/downloads/submissions/58257/versions/2/screenshot.png}{https://se.mathworks.com/matlabcentral/mlc-downloads/downloads/submissions/58257/versions/2/screenshot.png}
    \label{fig:pid}
\end{figure}

Στη συγκεκριμένη εργασία, ο ελεγκτής PID χρησιμοποιείται για τον έλεγχο της θέσης του drone στο χώρο. Λαμβάνει ως είσοδο την θέση που βρίσκεται το drone και την επιθυμητή θέση και υπολογίζει την απόσταση τους. Η έξοδος του ελεγκτή υπολογίζεται με βάση τις τιμές των τριών όρων που αναλύθηκαν προηγουμένως και μετασχηματίζεται στη συνέχεια στην ταχύτητα που πρέπει να δοθεί στο drone, ώστε να φτάσει ομαλά στην επιθυμητή θέση. Χρησιμοποιούνται τέσσερις διαφορετικοί ελεγκτές, οι οποίοι υπολογίζουν ξεχωριστά το σφάλμα σε $x$, $y$, $z$ και $yaw$, καθώς είναι αδύνατο να βρεθεί μια διαμόρφωση που να καλύπτει όλες τις απαιτήσεις του ελεγκτή. Συνεπώς, ρυθμίζουμε ξεχωριστά την γραμμική ταχύτητα στον $x$, $y$ και $z$ άξονα και την περιστροφική ταχύτητα γύρω από τον $z$ άξονα, δηλαδή την κίνηση $yaw$. Για να εξασφαλίσουμε την πιο ομαλή μετάβαση του ρομπότ στην επιθυμητή θέση, εφαρμόζουμε έναν διαδοχικό έλεγχο των τεσσάρων μεταβλητών. Πιο συγκεκριμένα, πρώτα ρυθμίζεται το ύψος και η θέση $(x,y)$ του drone και τέλος η κατεύθυνση του.

Για τον έλεγχο της θέσης, οι τρεις όροι επηρεάζουν με τους παρακάτω τρόπους την έξοδο του ελεγκτή και συνεπώς την θέση του drone: 
\begin{itemize}
    \item {Αναλογικός: Όσο πιο μεγάλη είναι η απόσταση του drone από την επιθυμητή θέση, τόσο μεγαλύτερη είναι και η επίδραση του όρου αυτού. Αντίθετα, αν πλησιάζει στο στόχο η επίδραση μειώνεται.}
    \item {Ολοκληρωτικός: Εάν το σφάλμα είναι μικρό για μεγάλο χρονικό διάστημα ή υψηλό για σύντομο χρονικό διάστημα αυξάνεται η επίδραση αυτού του όρου. Αντίθετα, αν το σφάλμα έχει μία μέτρια τιμή, περιορίζεται η επίδραση του. Ο όρος αυτός είναι πολύ σημαντικός για τα UAVs σε περιβάλλοντα με υψηλό αέρα, καθώς διατηρεί σταθερή τη θέση του από τις παρεμβολές. Καθώς η εργασία αυτή πραγματοποιείται εξ' ολοκλήρου σε περιβάλλον προσομοίωσης και δεν υπάρχουν τέτοιου είδους επιδράσεις, ο όρος αυτός αγνοείται.}
    \item {Παραγωγικός: Όσο αυξάνεται το σφάλμα, αυξάνεται και η επίδραση του όρου, ενώ όσο μειώνεται το σφάλμα παράγει αρνητική έξοδο, ώστε να μειωθεί η περίπτωση υπερύψωσης από τον αναλογικό όρο. Σε περίπτωση που δεν μεταβάλλεται το σφάλμα, ο όρος αυτός δεν επιδρά καθόλου. Μεγάλες τιμές του όρου αυτού μπορούν να οδηγήσουν σε αστάθεια και δονήσεις του drone, λόγω των απότομων μεταβολών.}
\end{itemize}
