\section{Hector Quadrotor}
\label{section:hector}

Το Hector Quadrotor\footnote{\href{http://wiki.ros.org/hector_quadrotor}{http://wiki.ros.org/hector\_quadrotor}} stack \cite{2012simpar_meyer} είναι ένα σύνολο ROS πακέτων που αφορούν την μοντελοποίηση, τον έλεγχο και την προσομοίωση σε Quadcopter. Η πρώτη του έκδοση δημοσιεύτηκε το 2012 από την ομάδα Team Hector του Πολυτεχνείου του Darmstadt.

Χρησιμοποιείται το περιβάλλον προσομοίωσης Gazebo\footnote{\href{http://gazebosim.org/}{http://gazebosim.org/}}, καθώς περιλαμβάνει όλους τους φυσικούς νόμους που υπάρχουν σε ένα πραγματικό περιβάλλον και ένα μεγάλο σύνολο ρομπότ, περιβάλλοντων και αισθητήρων που μπορούν να χρησιμοποιηθούν. Για την περιγραφή των χαρακτηριστικών του drone στον προσομοιωτή, χρησιμοποιείται ένα μοντέλο URDF\footnote{\href{http://wiki.ros.org/urdf/XML/model}{http://wiki.ros.org/urdf/XML/model}} που περιλαμβάνει σημαντικά χαρακτηριστικά, όπως η μάζα και η αδράνεια των τμημάτων του quadcopter. Επίσης μπορούν να προστεθούν ή και να μεταβληθούν με ευκολία οι αισθητήρες του.

Το συγκεκριμένο μοντέλο διαθέτει IMU, βαρομετρικό αισθητήρα για την προσομοίωση της στατικής πίεσης σε συγκεκριμένα υψομέτρα, αισθητήρα απόστασης sonar για την εκτίμηση του ύψους, αισθητήρα μαγνητικού πεδίου για τον υπολογισμό της διεύθυνσης του drone και τέλος δέκτη GPS. Εκτός αυτών, υπάρχει η επιλογή για την χρήση κάμερας και laser, ανάλογα με τις ανάγκες του περιβάλλοντος και της εφαρμογής.

\begin{figure}[!ht]
    \centering
    \includegraphics[width=0.8\textwidth]{./images/chapter3/hector.png}
    \caption{Το μοντέλο του Hector Quadrotor στον προσομοιωτή Gazebo} 
    \label{fig:hector}
\end{figure}

Το πακέτο αυτό χρησιμοποιήθηκε για την μοντελοποίηση του drone στην συγκεκριμένη διπλωματική εργασία, καθώς περιλαμβάνει όλα τα απαραίτητα κομμάτια κώδικα για την ρεαλιστική προσομοίωση του συστήματος. Συνεπώς, με την χρήση και την τροποποίηση αυτών, δόθηκε έμφαση στην επίλυση του προβλήματος και στη δημιουργία μιας λύσης που θα μπορεί να χρησιμοποιηθεί με οποιοδήποτε μοντέλο drone. 

Πιο συγκεκριμένα, μεταβλήθηκαν οι αισθητήρες του drone, ώστε να ταιριάζουν με τις απαιτήσεις του προβλήματος, ενώ χρησιμοποιήθηκαν αυτούσιοι οι αεροδυναμικοί παράμετροι και οι ελεγκτές που μεταφράζουν την ταχύτητα που δίνεται στους άξονες σε ομαλή κίνηση του αεροσκάφους.